\chapter{Complex clauses} 
\label{ChapterSub} 
\begin{sloppypar}
This chapter describes complex clauses in Vamale: coordinated and subordinated clauses, including relative clauses. This excludes fronting, which is discussed in \sectref{sec:fronting}, as it mostly concerns noun phrases. Subordination is a central part of Vamale syntax, since virtually every modification of a verb's argument happens through a subordinated clause, as does much verbal modification. Relative clauses are discussed in \sectref{sec:RelCl}. Like most Oceanic languages, Vamale does not have a pivot function of the subject; instead it marks the subject in all coordinated and subordinated clauses, whether the subordinate subject is the same as the one in the matrix clause or not. %does not have a pivot function of the subject\todo{Rephrase or delete this please, it does not make sense to me}, i.e. t
The object of a matrix clause cannot be the implicit subject of a subordinate clause (e.g. \textit{I\textsubscript{i} saw the man who \_\textsubscript{i} walked past.}), nor does the subject of one coordinated clause automatically have the same referent as the omitted subject of the other coordinated clause (e.g. \textit{I\textsubscript{i} come to \_\textsubscript{i} work.}). 
Usually, a subject is present in every clause, with the exception of adverbial clauses (\sectref{sec:adv_can}). 
\end{sloppypar}

Some morphemes have both subordinating and coordinating functions, e.g. \textit{ma} (\sectref{ssec:ma}, \sectref{ssec:cond_ma}), and the presence of insubordination, usually with adhortative or optative function, confronts us with matrix and subordinate clauses that look the same. Insubordinate clauses are structurally identical to subordinate clauses, with the exception that they do not depend on a matrix clause (\sectref{ssec:Insub}). Apart from complement clauses, which are unambiguous, there are cases of two subordinate clauses coordinated by a conjunction, showing that they do not occupy the same slot (ex. \ref{ex:hapi_ma} in \sectref{ssec:hapi}). Furthermore, coordinate clauses cannot be fronted (though a coordinating conjunction may be the first element of a clause), whereas some subordinate ones can. 

Since most coordinators, subordinators, and the relativizer all end in /a/, the 3\gl{sg} subject index \textit{a} is not pronounced separately. All other subject indices do occur, albeit often in a fused form with the preceding morpheme, e.g. \textit{m=ase} for \textit{ma gase} \qu{\gl{subr} 1\gl{pl}.\gl{incl}}, or \textit{m=e} for \textit{ma e} \qu{\gl{subr} 1\gl{sg}}. 

Coordination reduction constructions, in the sense that subjects are omitted from consecutive coordinated clauses if they are the same, exist in Vamale (\ref{ex:co_ma}), contrary to the typical Oceanic pattern \parencite[517]{ross_morphosyntactic_2004}. 
The same is the case for subordinate clauses, though an absence of subject means a more immediate sequence: compare (\ref{ex:sub_red}), where killing is the underlying purpose of the matrix verb, to (\ref{ex:ma_silaa1}), where raising children is a more general and long-term goal not immediately tied to going into the house.

\ea \label{ex:sub_red}
\gll le=hma wati-le ma xaa-le hnuuda-me cahni\\
 3\gl{pl}=arrive chase-3\gl{obj} \gl{subr} beat-3\gl{pl}.\gl{obj} upstream-\gl{dir.cp} here\\
\glt \qu{They came to hunt them in order to strike them coming here up the river.} {[HC1:7]}
 \z


\ea\label{ex:ma_silaa1}\gll ma le=hma-san a=feanake si-le joakan juu-mwa, {vwa} can vi ``In Fwe ta ca i=juu-mwa-ca ma go=silaa mu=nyai-m."\\
 when 3\gl{pl}=arrive-go 3\gl{sg}=show \gl{ben}-3\gl{pl} big real-house do \gl{adv}.\gl{subr} say Bark Figtree go.up in \gl{def}.\gl{sg}=real-house-\gl{prox} \gl{subr} 2\gl{sg}=raise \gl{def}.\gl{du}=child-2\gl{sg}.\gl{poss}\\
\glt \qu{When they arrived, he showed them a big house, and said while doing so ``Figtree-Bark, go up into the house, so as to raise your children."} {[GC 57.1]}
\z

To give an overview of a complex sentence, (\ref{ex:tree2}) shows a clause containing an equative clause, a relative clause, a verb phrase with an argument, and a second relative clause modifying the argument. 


\begin{figure}
\begin{forest}
		for tree={
			if n children=0{
				font=\itshape,
				tier=terminal,
			}{},
		}
		[S 
		[NP
		[ASS	[tha]	]
		[PN	[le]	]
		]
		[NP
		[Det	[li{=}]]
		[REL-S
		[REL
		[a]]
		[VP
		[PN
		[le{=}]
		]
		%	[Adv		[xhwat]] 
		[VP	
		[V
		[xhwi] 	]			
		[NP
		[Det
		[li{=}] ]
		[N[nyu]] 
		[REL-S [REL[a] ] [V[xhopwen]]]
		]
		[NP[ka][PN[le]]]
		]
		]		
		]	
		]		
		]		
	\end{forest}
	\caption{Tree-diagram of the sentence structure of example (\ref{ex:tree2}), with \textit{le} \qu{3\gl{pl}} instead of \textit{li apuli-aen} \qu{these people}}
	\label{fig:WOTree}
\end{figure}


\ea
\label{ex:tree2}
%\a
\gll tha le li=a= le=xhwat xhwi li=nyu a= xhopwen ka li=apuli-aen\\
 \gl{ass}  3\gl{pl} \gl{def}.\gl{pl}=\gl{rel}= 3\gl{pl}=almost eat \gl{def}.\gl{pl}=fish \gl{rel}= big \gl{sbj} \gl{def}.\gl{pl}=person-\gl{dem}.\gl{dist}\\
\glt \qu{These people are the ones who almost eat/ate the big fish.} 
\z

\section{Coordination}
\is{Coordination}
Coordination of clauses in Vamale holds no deep secrets. The string of clauses is articulated via elements which, in many cases, also appear in the coordination of noun and verb phrases. Both clauses maintain the same word order as single clauses. Coordination reduction constructions which would remove subject marking, are rare. One case of this is (\ref{ex:co_ma}) below.

\subsection{Comitative \textit{ma}}
\label{ssec:coord_ma}
\is{Coordination!Coordinator \textit{ma}}
Contrary to most other coordinators, \textit{ma} is not attested linking complete clauses, both with their own subject marking, TAM contour, etc. Instead, \textit{ma} is a common coordinator for noun phrases, and is found with verb phrases as well. As for noun phrases, the clauses linked must be semantically part of the same group (\ref{ex:co_ma}). I mention \textit{ma} here because \textit{ma vwa-tau} is a coordination reduction construction: as the subject and the TAM contour are the same, indeed the verbs are semantically close (hunt and fish), \textit{vwa-tau} \qu{fish} does not need any of the usual particles surrounding it.

\ea \label{ex:co_ma}\gll go=caihnan naen goon ma abu=ja vap {ma} vwa-tau, pa ja tehnang li=si-bu\\
 2\gl{sg}=know now enough \gl{subr} 1\gl{du}.\gl{excl}=\gl{prf} hunt \gl{cnj} do-impact \gl{prf} \gl{prf} sharp \gl{def}.\gl{pl}=hand-1\gl{du}.\gl{excl}.\gl{poss}\\
\glt \qu{You know, now we can hunt and fish, our hands have become deft.} {[GC:89]}
\z

\subsection{\textit{kona}, \textit{kon} \qu{then}}
\is{Coordination!Coordinator \textit{kon(a)}}
The classic coordinators \textit{kona} (\ref{ex:kona}) and \textit{kon} (\ref{ex:kon}) both mean \qu{so, and then}, though \textit{ko na} \qu{and/but \gl{dem}} also exists, identical in syntactic distribution and phonological shape. \textit{kona} and \textit{kon} are semantically close and present a similar form, and may in fact be allomorphs or free variants. 


\ea\label{ex:kona}\gll kona tho-vi nyako i se Ngeein ka i se Cada\\
 \gl{cnj} call-say \gl{obl} \gl{def}.\gl{sg}=other N. \gl{cnj} \gl{def}.\gl{sg}=other C.\\
\glt \qu{And he called the one of them Sound and the other Beat.} {[GC:551]}
\z


\ea\label{ex:kon}
\gll kon abe=kon xaahni ca=piece-abe ma abe=thake-bigo\\
 \gl{cnj} 1\gl{pl}.\gl{excl}=\gl{prog} watch \gl{indf}.\gl{pl}=coin-1\gl{pl}.\gl{excl} \gl{subr} 1\gl{pl}.\gl{excl}=throw-bingo\\
\glt \qu{And we're saving some money for ourselves to play bingo.} {[AG1:67]}
\z

\subsection{\textit{hai} \qu{or}}
\is{Coordination!Coordinator \textit{hai}}
The coordinator \textit{hai} is used in three contexts: to coordinate two noun phrases or clauses (introduced in \sectref{ssec:cntr_hai}), to close a clause, implying that a list of possibilities goes on (\ref{ex:final_a}), and to begin a clause that is in direct contrast with the preceding one (\ref{ex:hai_first}). In most unmarked scenarios, \textit{hai} is realised {[a]}. The \textit{hai} form cannot be substituted by \textit{a} when contrasting two noun phrases \textit{hai X hai Y?} \qu{X or Y?} (see \sectref{ssec:cntr_hai}), but is otherwise an allomorph of \textit{a}. Note that in (\ref{ex:hai_ma}), \textit{hai} coordinates two subordinate clauses introduced by \textit{ma}.


\ea\label{ex:final_a}
\gll naen mwa a xahmaen mwa a\\
 now \gl{rep} \gl{cnj} tomorrow \gl{rep} \gl{cnj}\\
\glt \qu{See you later, or tomorrow, or\ldots}
\z

\ea\label{ex:hai_first}\gll xhaohmu tha juu the-\textit{profiter} tha juu \textit{facturé} mu=\textit{palette} \textit{agglo} yayo. a i=\textit{camion} \textit{choc}, ya tha \textit{choc}\\
 old \gl{ass} real \textsc{the\textsubscript{punct}}-profit \gl{ass} real write.up \gl{def}.\gl{du}=palette cement.brick \gl{expl} \gl{cnj} \gl{def}.\gl{sg}=truck good 3\gl{sg} \gl{ass} good\\
\glt \qu{The old man, he really made a quick profit, he made them pay for the two cement brick palettes, damn. The truck was fine however, it was fine.} {[KG:535-536]}
\z


\ea\label{ex:hai_ma}\gll ca i=wadan-aen {cahma} Xa-xhwi Apuli, a=nyawân aman ka apuli, ma jeena-n meekan ka a=tena i=a= le=kon e-vi nya pwa i=bwan, a=kon e-hnyimake ma a=xhwii-le {hai} {ma} a=cee-le ma le=han.\\
 in \gl{def}.\gl{sg}=time-\gl{dem} \gl{top} \gl{agt}.\gl{nmlz}-eat person 3\gl{sg}=spirit thing \gl{cnj} person \gl{com} ear-3\gl{sg}.\gl{poss} everywhere \gl{cnj} 3\gl{sg}=hear \gl{def}.\gl{sg}=\gl{rel}= 3\gl{pl}=\gl{prog} \gl{recp}-say put on \gl{def}.\gl{sg}=mountain 3\gl{sg}=\gl{prog} \gl{refl}-think \gl{subr} 3\gl{sg}=eat-3\gl{pl} \gl{cnj} \gl{subr} 3\gl{sg}=leave-3\gl{pl} \gl{subr} 3\gl{pl}=go\\
\glt \qu{At this moment, Maneater, who was half man, half spirit and had ears everywhere, and had heard what was said on the mountain, wondered whether he was going to eat them or let them go.} {[GC:107]}
\z

\subsection{Contrastive \textit{ka}}
\is{Coordination!Coordinator \textit{ka}}
Like for noun phrases, a clause introduced in contrast to the preceding one is usually coordinated with \textit{ka}.

\ea\gll tha le=thagavi yee ko-n gi gi ka thala a= xhopwen {ka} gaa naen vwa meekan nyasi-je vwa \textit{tronçonneuse}\\
 \gl{ass} 3\gl{pl}=cut tree \gl{obl}-\gl{nspec} axe axe \gl{cnj} knife \gl{rel}= big \gl{cnj} 1\gl{sg}.\gl{incl} now \gl{exist} everything for-1\gl{incl}.\gl{poss} \gl{exist} chainsaw\\
\glt \qu{They cut trees by axe, axe and machete, but we now, we have everything, we have chainsaws.} {[KP:103]}
\z


\subsection{\textit{ko} \qu{but}}
\label{sec:conj_ko}
\is{Coordination!Coordinator \textit{ko}}
\textit{ko} \qu{on} is one of the most versatile particles in the language, i.e. \textit{ko} appears with many different functions. Apart from a preposition \qu{on}, and an oblique marker, \textit{ko} is also used on an interclausal level, ranging from the coordinating \qu{and, but} (\ref{ex:ko_but1}) and (\ref{ex:ko_but2}), to the subordinating, adjunct-adding \qu{because} (\ref{ex:ko_reason}).


\ea\label{ex:ko_but1}\gll ko na kai a= eca-kau ko nien-aen?\\
 \gl{cnj} \gl{dem} who \gl{rel}= teach-2\gl{du}.\gl{obj} \gl{obl} \gl{dem}.\gl{pl}-\gl{dist}\\
\glt \qu{But who is it that taught you two about all this?} {[GC:81]}
\z

\ea\label{ex:ko_but2}\gll ko le, li=xhaohmu, tha xahnang-le\\
 \gl{cnj} 3\gl{pl} \gl{def}.\gl{pl}=old \gl{ass} good-3\gl{pl}\\
\glt \qu{But they, the elders, they were fine.} {[RP:16]}
\z

\ea\label{ex:ko_reason}\gll e=vi ko e=bwa xaleke\\
 1\gl{sg}=say because 1\gl{sg}=\gl{ipfv} see\\
\glt \qu{I'm saying this because I still got to see it.} {[KL:114]}
\z


\subsection{\textit{koin} \qu{then}}
\is{Coordination!Coordinator \textit{koin}}
\label{ssec:koin}
The coordinator \textit{koin} \qu{then} derives from the word \textit{koin} \qu{end}, which exists as a noun and as a verb (\ref{ex:koin}). 

\ea  \label{ex:koin}\gll {koin} hut pwan maa hma-cu xahut ka cama i=khîî a=han moo ko-n maa\\
 end go.down on reef arrive-stand down.there \gl{cntr} \gl{top} \gl{def}.\gl{sg}=swamp.hen 3\gl{sg}=walk stay on-3\gl{sg}.\gl{poss} reef\\
\glt \qu{When the tide had well receded, the swamp hen went walking [fishing] on the reef.} {[HC2:1]}
\z

\ea\gll a=vi maman Henri ma maama-le-mwa lu=tua a=fe i=dipi ka i=see {koin} a=fe sanan ka i=see \\
 3\gl{sg}=say mother H. \gl{com} mother-3\gl{pl}.\gl{poss}-\gl{deict} 3\gl{du}=take.out 3\gl{sg}=take \gl{def}.\gl{sg}=cover \gl{sbj} \gl{def}.\gl{sg}=other finish 3\gl{sg}=take content \gl{sbj} \gl{def}.\gl{sg}=other \\
\glt  \qu{That is, Henri's mother and the mother of the others, the two unwrapped it and one took the cover [of the War Money], then the other took the content.} {[HC1:17]} 
\z
%\a
%
%\ili{}{}{} HC2
%\gll koin a=kon a=bwa tena ka i=ibwen
%
% then 3\gl{sg}=\gl{koon} 3\gl{sg}=\gl{bwa} hear \gl{sbj} \gl{def}.\gl{sg}=squid
%
%\glt \qu{Then he was- The squid heard him [crying]}
%
%
\ea
%\ili{}{}{}  
\gll vwa i=fwa koo-n. koin i=mapu le=mu ta can \\
 \gl{exist} \gl{def}.\gl{sg}=hole on-3\gl{sg}.\gl{poss} after.that \gl{def}.\gl{sg}=bee 3\gl{pl}=\gl{freq} go.up in\\
\glt \qu{There is a hole in [the tree]. And the bee, they will go up in there.} {[KM:2-3]}
\z

If \textit{koin} occurs before a VP, it seems to mean ``while", as in (\ref{ex:koin_while}). 

\ea \label{ex:koin_while}\gll a=kon vi hapi na gasu=\textit{enregistrer} koin gase=bo pala \\
 3\gl{sg}=\gl{prog} say that \gl{dem} 1\gl{du}.\gl{incl}=record while 1\gl{pl}.\gl{incl}=\gl{irr} say \\
\glt  \qu{He is saying that we will record while we will be speaking.} {[Tipije:1]}
\z 

\subsection{Contrastive \textit{kavi}}
\is{Coordination!Coordinator \textit{kavi}}
\textit{kavi} \qu{but} introduces a coordinate clause that contrasts strongly with the preceding one (\ref{ex:kavi}). While it most often coordinates two clauses, \textit{kavi} is also attested at the beginning of an utterance, to contrast it with statements made immediately before in the conversation (\ref{ex:kavi2}).

\ea \label{ex:kavi}\gll  cip=abe=vwa-taeke {kavi} abe=vwa \\
  \gl{neg}=1\gl{pl}.\gl{excl}=do-badly but 1\gl{pl}.\gl{excl}=do\\
\glt \qu{We're not doing [custom] badly, we're (simply) doing [the work].} {[CP1:23]}
\z

\ea\label{ex:kavi2}
%\ili{}{}{}  
\gll kavi th=abe vwa nyeca i=teete \\
 but \gl{ass}=1\gl{pl}.\gl{excl} do in \gl{def}.\gl{sg}=aunty\\
\glt \qu{But we're doing [the funerary work] with the [deceased] Aunty in mind.} {[CP1:24]}
\z

\section{Subordination}
\label{sec:subr}
\is{Subordination}
Most subordinate clauses, apart from complement and relative ones, are introduced by the neutral verbal subordinator \textit{ma}, as in (\ref{ex:ma}), or by a complex form containing it, e.g. \textit{cama} \qu{if (\gl{irr})} and \textit{ko-ma} \qu{so that (lit. because-\gl{subr})}, see (\ref{ex:koma}).


\ea\label{ex:ma}\gll ju-vaa vwasoon ma gase=vwa li=vaaya-n li=xhaohmu\\
 too.much impossible \gl{subr} 1\gl{pl}.\gl{incl}=do \gl{def}.\gl{pl}=work-\gl{poss} \gl{def}.\gl{pl}=old\\
\glt \qu{We cannot do the works of the elders.} {[KP:98]}
\z


\ea\label{ex:koma}\gll udu li=fati li=xhaohmu ko-ma e-vwa ka-n nyakoo-m ca i=thoatit a= bwa la\\
 drink \gl{def}.\gl{pl}=word \gl{def}.\gl{pl}=old because-\gl{subr} \gl{ins}.\gl{nmlz}-do \gl{abs}-\gl{nspec} for-2\gl{sg}.\gl{poss} in \gl{def}.\gl{sg}=day \gl{rel}= \gl{ipfv} be.here\\
\glt \qu{Drink the words of the elders so that they be tools for you in the day that will come.} {[GD:2]}
\z



\subsection{Complementation}
\is{Subordination!Complementation}
\is{Complementation}
Complement clauses are introduced with \textit{hapi (na)} for verbs of perception and locution (e.g. \textit{vii} \qu{say}, \textit{tena} \qu{hear}), and with \textit{ma} for modal verbs. 

\subsubsection{Modal complementizer \textit{ma}}
\label{ssec:ma}
\is{Complementation!Complementizer \textit{ma}}

Modal constructions in Vamale are to a large extent formed in the same way: a matrix clause consisting of a modal word, and the proposition modified by the modal word, i.e. a complement clause. While the complement clause usually takes subject index marking, the 3\gl{sg} form \textit{a} merges with the subordinator \textit{ma} and is used in contexts where the subject is unknown. In (\ref{ex:comp_ma}), the modal word is the stative verb \textit{xahnang} \qu{good}, which introduces the desirable proposition (\qu{if I knew what you are talking about}).
%possessible verb \textit{nyima-} \qu{to want, to like}, which is derived from the inalienable noun \textit{nyima-} \qu{heart, will}. The vowel of the complementizer \textit{ma} assimilates to the subject marker of the complement clause, \textit{e}.

\ea \label{ex:comp_ma}
\gll xahnang m=e bo caihna-n hapi go=pala ko i=da\\
 good \gl{comp}=1\gl{sg} \gl{irr} know-\gl{ana} \gl{comp} 2\gl{sg}=speak \gl{obl} \gl{def}.\gl{sg}=what\\
\glt \qu{It would be good if I knew what you are talking about.}
\z

Modal words include in fact only two non-verbs: the deontic modal word \textit{goon} \qu{enough} $\rightarrow$ \textit{goon ma V...} \qu{it is possible, it is allowed}, and \textit{xhwan} \qu{bite, bit} $\rightarrow$ \textit{xhwan ma V...} \qu{barely, almost} (\ref{ex:xhwan e-goakan}). The other members are verbs. Some verbs are decategorialized and also exist as stative verbs with predicative function, e.g. \textit{xahnang} \qu{good} and \textit{nyau} \qu{bad}, see \textit{xahnang-eo} \qu{I am good}.

Other modal verbs, especially impersonal ones, always have a modal function, and always take a complement. This includes the deontic verbs \textit{vwasoon} \qu{impossible} and \textit{siteke} \qu{taboo}, as well as epistemic \textit{vaang}\is{Verbs!Impersonal verbs} \qu{unknown}. Like other epistemic modal verbs, \textit{vaang} may take the complementizer \textit{hapi} as well as \textit{ma}.
A subset of these modal verbs inflects for person: \textit{nyima-n ma} \qu{s/he wants, that...}, \textit{saxhwe-a ma} \qu{to not want, that}, \textit{sahnaang-ea ma} \qu{to not understand if}, \textit{cacahniing-ea ma} \qu{to be unsure if}.

\ea \label{ex:xhwan e-goakan}\gll yo, xhwan e-goakan se m=e=bwa xhwi nyu\\
 1\gl{sg} hardly \gl{mid}-time one \gl{subr}=1\gl{sg}=\gl{ipfv} eat fish\\
\glt \qu{I rarely eat fish (Hardly is there a time when I eat fish).} {[X10:24]}
\z

\subsubsection{Complementizer \textit{hapi}}
\label{ssec:hapi}
The complementizer \textit{hapi} is used to introduce the argument of verbs of ``locution or perception" \parencite[53]{lynch_oceanic_2002}, e.g. \textit{vii} \qu{say}, \textit{hnyimake} \qu{think}, etc. Coupled with \textit{na} \qu{\gl{dem}}, \textit{hapi} can also introduce quoted speech as in \textit{a vi hapi na} \qu{s/he says that}. 

\ea\label{ex:hapi}\gll In-Fwe hapi ``In-Fwe ka e=hu-pe nya hnya-da xa-da" kavi cipa a=vi i=goakan.\\
 F. \gl{comp} F. \gl{cnj} 1\gl{sg}=come-\gl{dir.cp} from \gl{prox}-move.up \gl{loc}.\gl{adv}-move.up but \gl{neg} 3\gl{sg}=say \gl{def}.\gl{sg}=place\\
\glt \qu{Figtree-Bark said that ``[my name is] Figtree-Bark and I come from somewhere a little further up" but she didn't say the place.} {[GC:53.2]}
\z

However, \textit{na} can be omitted (\ref{ex:hapi}), as can even \textit{hapi}, in a context where the clause boundaries are otherwise clarified (\ref{ex:no hapi}). In the latter example, the main verb is \textit{fwajimwa-ke} \qu{ask-\gl{tr}}, which is transitive but demands an Experiencer argument (the person asked). The other clause cannot be added with \textit{hapi}, as the latter is governed by a semantically defined group of verbs, which does not include \textit{fwajimwake}. Hence we have two clauses: matrix clause \textit{e fwajimwako} and matrix clause \textit{kai a i vukin-ea}. \textit{i vukin-ea}, a nominalized form of the stative verb \textit{vukin-} \qu{to be the cause}, is the predicate of the equative clause with \textit{kai}, which is itself an content of the question \textit{fwajimwa-ko}, but cannot be added to the main verb. The two clauses are distinguishable by prosody, as \textit{kai} marks the beginning of a new pitch contour. Note that \textit{kai} is fronted here, as it is the focussed element of the question; the unmarked order would be \textit{i vukin-ea kai?}.

\ea \label{ex:no hapi}\gll e=fwajimwa-ko: ``kai a i=vuki-n-ea"\\
 1\gl{sg}=ask-2\gl{sg} who 3\gl{sg} \gl{def}=cause-\gl{poss}-3\gl{sg}.\gl{S\textsubscript{P}}\\
\glt \qu{I ask you who the culprit [of this] is.} {[D3:110]} 
\z
%one can see because there is no indirect argument marker \textit{nyako}.
The complementizer \textit{hapi} is not necessarily the only morpheme to subordinate a clause. In (\ref{ex:hapi_ma}), the speaker is more certain of the content of the irrealis subordinate clause than in the unmarked example (\ref{ex:hapi_a}). 


\ea\label{ex:hapi_a}
\gll e=caihnan hapi a=bo vwa\\
 1\gl{sg}=know \gl{comp} 3\gl{sg}=\gl{irr} do\\
\glt \qu{I know that he will do it.}
\z


\ea\label{ex:hapi_ma}
\gll e=caihnan hapi ma a=bo vwa\\
 1\gl{sg}=know \gl{comp} \gl{subr} 3\gl{sg}=\gl{irr} do\\
\glt \qu{I know with certainty that he will do it.}
\z 
 
\subsection{Adverbial clauses with \textit{can}}
\label{sec:adv_can}
\is{Adverbs!Adverbial clause}
\is{Subordination!Adverbial clause}

Adverbial clauses are introduced with \textit{can}, derived from \textit{ca-n} \qu{in-\gl{nspec}} and are indistinguishable from matrix clauses, except that the former's verb does not take subject marking (\ref{ex:can}). Since the adverbial clause's subject must be the same as that of the matrix, the subject index proclitic is omitted. Aspect markers are rarely used, as the TAM contour is the same as the matrix clause's, but \textit{bwa} \qu{\gl{ipfv}} was overheard. Stative verbs that retain their subject marking are not attested. This weak desententialization is somewhat unusual for Oceanic languages, which canonically do not desententialize their adverbial clauses at all \parencite[519]{ross_morphosyntactic_2004}. 

\ea
\label{ex:can}\gll gase=xadaa ha-mwa ca-n sate-n moko i=hun-moo-gaa\\
 1\gl{sg}.\gl{incl}=on.the.other.hand go-\gl{deict} in-\gl{nspec} be.different-\gl{nspec} \gl{cpr} \gl{def}.\gl{sg}=\gl{nmlz}-be-1\gl{sg}.\gl{incl}\\
\glt \qu{We however walk now differently from our traditional ways.} {[2017-08-48.1]}
\z

\ea\gll go=han ca-n hnyimake thamo\\
 2\gl{sg}=go in-\gl{nspec} think.about woman\\
\glt \qu{You're walking while thinking of women.} {[G4 22.1]}
\z

If an adverbial clause comes after a verb's argument, especially if the resulting construction is long, a resumptive \textit{vwa} \qu{do} may be introduced as an anaphoric host to the adverbial clause (\ref{ex:ma_silaa}).

\ea \label{ex:ma_silaa}\gll ma le=hma-san a=feanake si-le joakan juu-mwa, {vwa} can vi ``In Fwe ta ca i=juu-mwa-ca ma go=silaa mu=nyai-m."\\
 when 3\gl{pl}=arrive-go 3\gl{sg}=show \gl{ben}-3\gl{pl} big real-house do \gl{adv}.\gl{subr} say Bark Figtree go.up in \gl{def}.\gl{sg}=real-house-\gl{prox} \gl{subr} 2\gl{sg}=raise \gl{def}.\gl{du}=child-2\gl{sg}.\gl{poss}\\
\glt \qu{When they arrived, he showed them a big house, and said while doing so ``Figtree-Bark, go up into the house, so as to raise your children."} {[GC 57.1]}
\z

\subsection{Relative clauses}
\label{sec:RelCl}
\is{Relative clause}

Relative clauses in Vamale are typical of Oceanic languages, in that they are introduced by a morpheme that looks like a pronoun \parencite[516]{ross_morphosyntactic_2004}. In our case, \textit{a} \qu{\gl{rel}} is formally identical to \textit{a} \qu{3\gl{sg}.\gl{S\textsubscript{A}}}. Relative clauses feature resumptive morphemes which allow the language to relativize a noun phrase in any position on the Accessibility Hierarchy. All NPs can be represented by a resumptive morpheme, but this is not obligatory for the subject.

\ea
\gll e=xaleke i=xawakhan a= tana\\
 1\gl{sg}=see \gl{def}.\gl{sg}=dog \gl{rel}= red\\
\glt \qu{I see the red dog.} (lit. \qu{I see the dog that red.})
\z

%An argument of a verb, but also the predicate of a noun which is not the subject? xx.

\ea\gll le=vwa ma le=thabilo li=a= le=fee-ko\\
 3\gl{sg}=do \gl{subr} 3\gl{sg}=kill \gl{def}.\gl{pl}=\gl{rel}= 3\gl{pl}=take-2\gl{sg}.\gl{obj}\\
\glt \qu{They will kill those who took you.} {[B1:8]}
%translated sentence
\z 
%%todo{source for vamale example}

If the relativized, inanimate noun phrase was already mentioned, it does not get mentioned again (\ref{ex:rel_no_rep}), unlike in Bwatoo (\ref{ex:rel_bwatoo}), where it reappears in the slot of its new syntactic function.

%%todo{vamale example}
\ea
\label{ex:rel_bwatoo}
\ort{zho tahmake ani meata a go thaxhuti-a} (Bwatoo)\\
\gll ðo tam̥ake anĩ mẽãta a ᵑgo θaxuti-a\\
 1\gl{sg} know \gl{def} story \gl{rel} 2\gl{sg} tell-3\gl{sg}.\gl{obj}\\
\glt \qu{I know the story that you are telling.} \citep[71]{rivierre_bwatoo_2006}
\z

%:71
\ea\label{ex:rel_no_rep}
\gll  e=holeke nya-si-m li=fati a= go=vi\\
 1\gl{sg}=thank put-hand-2\gl{sg}.\gl{poss} \gl{def}.\gl{pl}=word \gl{rel}= 2\gl{sg}=say\\
\glt \qu{I thank you for the words you said.} {[B2:94]}
\z

%%todo{whither this?} 
	
A relativized subject noun phrase which is the subject in the relative clause as well, is most often indexed on the subordinated verb, but the relativizer itself is commonly skipped (\ref{ex:tuu_mapu}).

\ea  \label{ex:tuu_mapu}
%\ili{}{}{}  
\gll tha vwa li=personnes le=caihnan tuu mapu\\
 \gl{ass} there.is \gl{def}.\gl{pl}=person 3\gl{sg}=know pull bee\\
\glt \qu{There are people who know how to pull bees [= harvest honey].} {[KM:13]}
\z

Several relative clauses may follow one another, modifying the same noun phrase (\ref{ex:rs_rs}), but this was not attested outside elicitations.

\ea \label{ex:rs_rs}
\gll e=holeke nya-si-m i=xawakhan {\ob}{a}= siim-ea{\cb} {\ob}{a}= go=nya-a{\cb} \\
 1\gl{sg}=thank put-hand-2\gl{sg}.\gl{poss} \gl{def}=dog \gl{rel}= mange-3\gl{sg} \gl{rel}= 2\gl{sg}=give-3\gl{sg}.\gl{obj}\\
\glt \qu{I thank you for the mangy dog you gave [me].} (lit. \qu{I give.thank.for to-you the dog that mangy-he that you-give-it})
\z

\subsection{Purposive function of \textit{ma}}
\is{Subordination!Purposive \textit{ma}}
A very common function of \textit{ma} is to subordinate a purposive clause (\ref{ex:ma1}). This can be used with causative meanings, like in \textit{vwa, ma \ldots} \qu{do, so that...} (see \sectref{sec:caus}), or alone. 

\ea \label{ex:ma1}
\gll a=nya s-ung m=e nya si-m\\
 3\gl{sg}=give \gl{ben}-1\gl{sg}.\gl{poss}	\gl{subr}=1\gl{sg}	give \gl{ben}-2\gl{sg}.\gl{poss}\\
\glt \qu{He gave it to me so that I would give it to you.}
\z

\subsection{Conditional \textit{ma}, \textit{cama} \qu{if}}
\label{ssec:cond_ma}
\is{Subordination!Conditional \textit{ma}}

When preceding a matrix clause, \textit{ma} and its derived form \textit{cama} introduce a hypothetical situation (\ref{ex:ma3}, \ref{ex:ma4}). As can be seen in (\ref{ex:cama1}), \textit{ma} and \textit{cama} can be used interchangeably in some contexts. As far as a difference could be made out between \textit{ma} and \textit{cama}, \textit{cama} seems to precede more markedly irrealis situations (\ref{ex:cama2}), whereas \textit{ma} is also used to refer to traditions (\ref{ex:ma2}). This \textit{ma} is probably the base form from which the insubordinator \textit{ma} was derived (see \sectref{ssec:Insub}).


\ea\label{ex:ma3}\gll gase=cahu ma bwa vwa wadan {ma} le=tho nyakoo-je ka li=xhaohmu\\
 1\gl{pl}.\gl{incl}=answer while \gl{ipfv} \gl{exist} time when 3\gl{pl}=call for-1\gl{pl}.\gl{incl}.\gl{poss} \gl{sbj} \gl{def}.\gl{pl}=elder\\
\glt \qu{We answer while there is still time, when(ever) the elders call on us.} {[GS:1-2]}
\z

\ea\label{ex:ma2}\gll {m}=abe bwa vwa \textit{nettoyage} h=abe thai li=\textit{vaisselle}-ea\\
 \gl{subr}=1\gl{pl}.\gl{excl} \gl{ipfv} do cleaning \gl{top}.\gl{rep}=1\gl{pl}.\gl{excl} pick.up \gl{def}.\gl{pl}=dish-3\gl{sg}.\gl{poss}\\
\glt \qu{When we do the clean up [at a wedding] we pick up [the bride's] dishes.} {[AG1:428]}
\z

\ea\label{ex:ma4}\gll Ma tha vwa eca=loto-n-gaa ma gase=ta can Wanaa\\
 \gl{cond} \gl{ass} exist \gl{indf}=car-\gl{poss}-1\gl{pl} \gl{subr} 1\gl{pl}.\gl{incl}=go.up in W.\\
\glt \qu{If we had a car we would go up to Wanaa.} {[D3:90]}
\z

\ea\label{ex:cama1}\gll cama	ha-mwa-me	ka	i=khîî	ma	hma-ca-mwa-me	paa	cika	mwa	wâng\\
 when	go-\gl{rep}-\gl{dir.cp}	\gl{sbj} \gl{def}.\gl{sg}=swamp.hen	when arrive-go.up-\gl{rep}-\gl{dir.cp} \gl{prf}	\gl{neg}.\gl{exist}	\gl{rep} boat\\
\glt \qu{When the swamp hen went back and arrived, there was no boat anymore [because the rat had eaten it].} {[HC2:5]}
\z 

\ea\label{ex:cama2}\gll ju hole ko calibeen gase=vwa cama vwa cama cika\\
 really thank because sometimes 1\gl{pl}.\gl{incl}=do \gl{subr} \gl{exist} \gl{subr} \gl{neg}.\gl{exist}\\
\glt \qu{Thank you very much, for it is our habit that we do [custom] whether we have [ceremonial goods] or not.} {[JU:7]}
\z

\subsection{Realis \qu{while}}
\label{ssec:while}
\is{Subordination!Realis \textit{cala} \qu{while}}

\textit{Cala} is syntactically identical to \textit{cama} and \textit{ma} (\ref{ex:cala}). Semantically, \textit{cala} is used only to introduce past situations firmly rooted in reality, and \textit{cama} for everything else.

\ea \label{ex:cala}\gll  ju vaa m=e juu saxhuti i=thuatit-abe cala abe=hut ko-n yeen nyeet\\
 real too \gl{subr}=1\gl{sg} real narrate \gl{def}.\gl{sg}=day-1\gl{pl}.\gl{excl}.\gl{poss} when 1\gl{pl}.\gl{excl}=go.down on-\gl{nspec} island when\\
\glt \qu{It's too much for me to properly tell the story of how when we went down the island the other day.} {[GP2:1]}
\z

\subsection{\textit{ko} \qu{because, thanks to}}
\label{sec:ko_cause}
\is{Subordination!\textit{ko} \qu{because}}
Subordinate clauses can also be linked to a matrix clause of which they are the cause. The subordinators used for this are \textit{ko} \qu{because} described in \sectref{sec:conj_ko}, as well as \textit{vuki-n} \qu{cause, stem}. The latter word acts both as a stative verb, when the reason is a participant themselves (\ref{ex:vukin-eong}), or as a subordinating particle, when the reason is an entire clause (\ref{ex:vukin}).

\ea\label{ex:vukin-eong}\gll  ‎‎kavi tha hmwaka-je, wanke mwa i=hun-moo ka i=bwanpu, hai vukin-gaa ko ko gase=juu vaaya xhayu a...\\
 \gl{cnj} \gl{ass} like-1\gl{pl}.\gl{incl} change \gl{deict} \gl{def}.\gl{sg}=\gl{nmlz}-stay \gl{abs} \gl{def}.\gl{sg}=land \gl{cnj} reason-1\gl{pl}.\gl{incl} because because 1\gl{pl}.\gl{incl}=real work random or \\
\glt \qu{But it's like us, changed the nature of the land or, we're the reason, because we just work at random [without custom], or\ldots} {[RP:9]}
\z

\ea\label{ex:vukin} \gll e=caihna-n vuki-n go=vi nyako-ong\\
 1\gl{sg}=know-\gl{ana} cause-\gl{nspec} 2\gl{sg}=say \gl{obl}-1\gl{sg}.\gl{poss}\\
\glt \qu{I know because you told me.} {[X9:3]} 
\z

\section{Insubordination}
\label{ssec:Insub}
\is{Insubordination}\is{Complementation!Complementizer \textit{ma}}
In order to mark a clause as adhortative (\ref{ex:adhort1}, \ref{ex:adhort2}), as a wish (\ref{ex:wish}), a regret (\ref{ex:signe}), or to diminish the speaker's assertiveness, the complementizer \textit{ma} can be put at the very beginning of a clause without a preceding matrix clause. 


\ea\label{ex:adhort1}
\gll ma gase=e-saam ko li=vaaya-n-gaa\\
 \gl{subr} 1\gl{incl}=\gl{recp}-help \gl{obl} \gl{def}.\gl{pl}=work-\gl{poss}-1\gl{incl}.\gl{poss}\\
\glt \qu{May we help each other in our works.} {[GB:5]}
\z


\ea\label{ex:adhort2}
\gll ma gavwe=xaleke, ka caihna-n\\
 \gl{subr} 2\gl{pl}=see \gl{cnj} know-\gl{ana}\\
\glt \qu{May you look [at this custom] and know (acknowledge) it.}
\z

\ea\label{ex:wish}\gll ma tha xa-vwa-wîîn m=e sam koo-m, ma gasu=vacuti i=juu.mwa-go\\
 \gl{cond} \gl{ass} \gl{nmlz}-\gl{exist}-strength \gl{subr}=1\gl{sg} help \gl{obl}-2\gl{sg} \gl{subr} 1\gl{du}.\gl{incl}=erect \gl{def}.\gl{sg}=hut-2\gl{sg}.\gl{poss}\\
\glt \qu{If only I were strong (enough) to I would help you, (so that) we build your house.} {[D3:91]}
\z

\ea \label{ex:signe}\gll \textit{putain}, yo, m=e th=e=koon ja vwa i=\textit{signe}\\
 fuck 1\gl{sg} \gl{subr}=1\gl{sg} \gl{ass}=1\gl{sg}=\gl{prog} \gl{prf} do \gl{def}.\gl{sg}=sign\\
\glt \qu{Dammit, I should have given [him] a sign by then.} {[KG:492]}
\z

Insubordination is important in deontic contexts, as it frames the predicate as something wished for by the speaker, or as their duty (see \sectref{ssec:ma}). There are no verbs meaning \qu{must}, \qu{may}, \qu{should} etc.
