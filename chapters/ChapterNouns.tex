% i want this section to say how i work and why, and what problems and particularities i encountered in the field. 
\chapter{Nouns} 
\label{ChapterNouns} 
\is{Nouns}

Vamale nouns are defined in this grammar as single words which can bear articles (see \sectref{sec:WCArticles}). Few other factors distinguish nouns from verbs, as nouns can be predicates with the same subject index markers as active verbs, see (\ref{ex:e-caacaa}). Only nouns, however, can be arguments of verbs, can be counted, can be possessed (though possessive morphology shows overlaps with some verbal morphology (see \sectref{sec:StatV} and \sectref{ssec:PossV}), and not all nouns can be possessed, e.g. \textit{jati} \qu{sea}). Although nouns do take some TAM marking, not all TAM marking is attested for nouns (e.g. \textit{bwa balan} \qu{only just (begun)}, \textit{kon} \qu{\gl{prog}}). Nouns are not inflected for number; this is covered by articles (for specific nouns; generic ones do not have articles).

	\ea 
	\label{ex:e-caacaa}
	\gll e=juura caacaa\\
	 1\gl{sg}=almost father\\
	\glt \qu{I am almost a father (soon).} {[B2:108]}
	 \z
	
	\ea
	\gll xhwat thuang m=e=caacaa\\
	 bit joke \gl{subr}=1\gl{sg}=father\\
	\glt \qu{I am almost a father (kind of).}
	\z

%Untergruppe: inalienable Nomina müssen besessen werden, im Notfall von 3\gl{sg} (falls -\textit{n} 3\gl{sg} darstellt, und 3\gl{sg} nicht -n-0 ist). Dies wird entweder mit den Possessivpronomen gekennzeichnet oder mit der Possessor-NP nach dem Verb.

\begin{sloppypar}
Vamale nouns can be classified along different dimensions. The animate\slash inanimate distinction, a semantic, lexically determined trait, affects their index\hyp marking on verbs. While index\hyp marking is treated in \sectref{ssec:TransV}, other effects of this distinction are described in \sectref{sec:animN}. Some nouns are uncountable (such as water, light, blood, etc.) and are thus not attested with non-singular articles. %and their syntactic origin (\Cref{sec:NomDeriv}). 
Finally, possessible nouns can be either alienably or inalienably possessed. This distinction is another typically Oceanic feature \parencite[41]{lynch_oceanic_2002}, though Vamale has added its own innovations (see \sectref{sec:Poss}). This chapter will briefly introduce these dimensions, but will focus only on possession, as the others are lexically determined. 
This chapter will also discuss classifiers (\sectref{sec:CL}). Vamale does not have many classifiers, and they are mostly relational: they add information about the nature of relationship between the possessor and the possessum. An exception are food classifiers, which, contrary to the possessive classifiers, can appear without the noun they classify. \sectref{ssec:NounCL} describes noun classifiers, which are obligatory in the context of plant species: \textit{mwago} \qu{mango} cannot appear alone; one must specify which part of the plant is meant. Noun classifiers are related to a much larger field of optional noun compound heads. Compound nouns are discussed in \sectref{sec:CompN}.
\end{sloppypar}

\section{Animacy}
\label{sec:animN}
\is{Animacy}

Nouns in Vamale are animate or inanimate. While other languages in New Caledonia distinguish human and non-human animate nouns (Cèmuhî even makes a difference between feminine and non-feminine nouns, \citealt[175]{rivierre_langue_1980}), this is of no importance in Vamale. An exception are some nouns, e.g. \textit{in maan} \qu{skin (human)} vs. \textit{in} \qu{skin (non-human, or dead human)}, and oblique markers:  \textit{nyasi-} \qu{\gl{ben}, \gl{top}} can only be used for humans whereas \textit{nyako-} is more general (see \sectref{sec:oblique}). %and the spatial prepositions \textit{puput} \qu{behind (inanim. subject)} and \textit{cai-} \qu{behind (anim. subject)} are for inanimate, and animate, nouns, respectively. 
Animate participants further trigger person marking on stative verbs (\ref{ex:animNstV}) and must be indexed with suffixes on transitive verbs (\ref{ex:animNOBJ}). Both contexts omit any indexing if the relevant noun phrase occurs within the verb phrase (\ref{ex:animNinVP}). 


	\ea\label{ex:animNstV}
	\gll ka abe niehni a= \textbf{thien-abe}\\	
	 \gl{cnj} 1\gl{pl}.\gl{excl} \gl{dem}.\gl{pl} \gl{rel}= three-1\gl{pl}.\gl{excl}\\
	\glt \qu{And we are those [masters of the rock], who are three (and we are these three masters of the rock).} {[DP:29]}
	\z
	
	
	\ea\label{ex:animNOBJ}
	\gll na cahni tha xhwan see-a a= thathe-\textbf{a}\\	
	 \gl{dem} here \gl{ass} a.bit one-3\gl{sg} \gl{rel}= kill-3\gl{sg}.\gl{obj}\\
	\glt \qu{Here there was only one that was killed.} {[HC1:22]}
	\z
	


\ea \label{ex:animNinVP}
(\textit{hmwet} is a stative verb)\\
\gll hmwet i=apuli\\
 tired \gl{def}.\gl{sg}=person\\
\glt \qu{The person is tired.} {[J8:7]}
\z

Another context in which animacy makes a difference is deverbal nominalizations, in which case the intransitive subject is only marked for person if the referent is animate: \textit{i hun-moo-a} \qu{\gl{def}.\gl{sg}=\gl{nmlz}-be-3\gl{sg}} \qu{his/her character}, but \textit{i hun-moo} \qu{its nature}. This is described in more detail in \sectref{sec:NomDeriv}.
\section{Possession}
\label{sec:Poss}
\is{Possession}

Possessed nouns make a distinction in alienability, i.e. whether they can occur without marking a possessor. This is a widespread Oceanic phenomenon \parencite[511]{ross_morphosyntactic_2004}. Alienable and inalienable nouns follow certain semantic tendencies outlined in Table \ref{tab:semantic_poss}, though there are numerous exceptions. Non-possessible nouns include proper names and unique concepts such as the sea or the sun, although poetic contexts may feature counterevidence. Another exception is \textit{la} \qu{place}, which can neither be generic (as would be indicated by \textit{-n} on preceding verbs and prepositions), nor take an article, but syntactically behaves like a noun otherwise (i.e. it follows prepositions) (\ref{ex:la}).

\ea \label{ex:la}
\gll suu cahni ca la\\
 break here in place\\
\glt \qu{Break it here at this spot!} {[KG:115]}
\z 

\begin{table}
	\caption{Semantic tendencies of possessed nouns}
	\begin{tabular}{ll}
		\lsptoprule
		Inalienable&	Alienable\\
		\midrule
		Body parts (except blood)&	Animals\\
		Things belonging to humans &	Plants\\
		(spirit, colour, appearance, strength)&\\
		Many kinship appelation terms (not the address forms)&	Tools\\
		\lspbottomrule
	\end{tabular}
\label{tab:semantic_poss}
\end{table}

\begin{sloppypar}
However, while most Oceanic languages distinguish direct (i.e. affixed) possession from indirect constructions using a relational classifier, Vamale has mostly done away with this distinction. Based on prosodic clues, especially stress shift, all possessive morphemes are considered suffixes, as in [ˈpu.a.ka] \qu{pig}, [pu.a.ˈka.ne.ɔŋ]\slash [pu.a.ˈka.ne.o] \qu{my pig}. The only clearly indirect possessive morpheme remaining is the linker \textit{ka-}, discussed in \sectref{kan}.
Possessive forms are not a sure sign of the nounhood of their host, even though nouns represent the vast majority of possessed lexemes. There are verbs with nominal morphology (e.g. \textit{hmana-n} \qu{hunger-3\gl{sg}.\gl{poss}, s/he is hungry}), and some stative verbs that have an identical nominal counterpart, e.g. \textit{mulip, muliv-ong} \qu{life, I am alive} (see \sectref{sec:WCVerbs} and \sectref{ssec:PossV}).
\end{sloppypar}

%\begin{table}
%	\parbox{0.45\linewidth}{
%		\centering
%		\caption{Alienable possessive suffixes (sometimes length changes: \textit{nya+ko+n} = \textit{nyakoon})}
%		\begin{tabular}{l|llll}
%			&	1&	1+&	2&	3\item
%			SG&		\multicolumn{2}{c}{eong}&		go&	ea\item
%			DU&	abu&	gasu&	gau&	lu\item
%			PL&	abe&	gase&	gavwe&	le\item
%		\end{tabular}
%	}
%	\hfill
%	\parbox{0.45\linewidth}{
%		\centering
%		\caption{Inalienable possession}
%		\begin{tabular}{l|llll}
%			&	1&	1+&	2&	3\item
%			SG&		\multicolumn{2}{c}{-ong}&		-m&	-n\item
%			DU&	-bu&	-ju&	-u&	-lu\item
%			PL&	-be&	-je / -ga &	-vwe&	-le\item
%		\end{tabular}
%	}
%\end{table}

There are several paradigms of possessive morphology, summarized in their most basic form in \Cref{tab:PossSuffix}. Paradigm I is mostly used for inalienable nouns, while paradigms Ib and II are used for alienable nouns. Loanwords exclusively take paradigm II forms.
%\todo{The table of pronouns should be given earlier with other pronouns, even if you analyse them here.}
\begin{table}
	\caption{Possessive suffix paradigms}
	
	\begin{tabular}{ll ccc}
		\lsptoprule
	    &    &   I& Ib& II \\
		\midrule
\gl{sg} &	1&	\textit{-ng}&	\textit{-ong}&	\textit{-eong}\\
		&	2&	\textit{-m}&	\textit{-am}&	\textit{-go}\\
		&	3&	\textit{-n}&	\textit{-an}&	\textit{-ea}\\
		\midrule
\gl{du} &	1\gl{incl}&	\textit{-ju}&	\textit{-aju}&	\textit{-gaeu}\\
		&	1\gl{excl}&	\textit{-bu}&	\textit{-abu}&	\textit{-abu}\\
		&	2&	\textit{-u}&	\textit{-au}&	\textit{-gau}\\
		&	3&	\textit{-lu}&	\textit{-alu}&	\textit{-lu}\\
		\midrule
\gl{pl} &	1\gl{incl}&	\textit{-je}&	\textit{-aje}&	\textit{-gaa}\\
		&	1\gl{excl}&	\textit{-be}&	\textit{-abe}&	\textit{-abe}\\
		&	2&	\textit{-vwe}&	\textit{-avwe}&	\textit{-gavwe}\\
		&	3&	\textit{-le}&	\textit{-ale}&	\textit{-le}\\
	    \lspbottomrule
	\end{tabular}
\label{tab:PossSuffix}
\end{table}

This basic overview given in \Cref{tab:PossSuffix} shows a distinction especially in the singular forms, where paradigm II has forms reminiscent of the free pronouns mentioned in \sectref{sec:WCPersPron}, and of the object suffixes discussed in \Cref{ChapterVerbs}, whereas paradigms I and Ib have forms that are unique. Indeed, paradigm Ib is the exact same as paradigm I, except that it inflects alienable forms, whose stems end in consonants. The similarity of paradigm II forms with pronouns could suggest that paradigms I and Ib have older morphology, and paradigm II was originally a possessive noun phrase, whose possessor NP was later incorporated (e.g. \textit{yee-n yo} \qu{tree-\gl{poss} 1\gl{sg}} $\rightarrow$ \textit{yee=n-eo} \qu{tree=\gl{poss}-1\gl{sg}}). The first person may have assimilated to paradigm I \textit{-(o)ng}. 

Contrary to paradigm I suffixes, paradigm II forms can attach to the end of noun phrases and of nominalized verb phrases (see \sectref{kan}). This difference in freedom of host selection is called ``direct" and ``indirect" possessive morphology, and many New Caledonian languages still oppose suffixes to free forms. For Vamale, I view \textit{-(e)ong} and the other paradigm II morphemes as suffixes, for at least 1\gl{sg} and 3\gl{sg} are different from the free pronominal forms that can be found in other possessive constructions, e.g. \textit{mama-n \textbf{gau} ma \textbf{yo}} \qu{the mother of you two, and me}, and the possessive forms are integrated into the stress structure (see \sectref{sec:Stress}).
%maahma-n,-m, maahma-ng

%Seeing that \textit{vap} &\qu{hunt} will rather be \textit{hun-vav-i-ka-n} &\qu{manner-hunt-epenthetc-?-\gl{poss}} or \textit{hunvap} than challenging the \textit{k} in \textit{kan}, \textit{k} is probably not negotiable.

\begin{sloppypar}
\citeauthor{hollyman_etudes_1999} identifies three main classes of possessed nouns in northern New Caledonian languages \parencite[61--62]{hollyman_etudes_1999}, listed below. While these are found in Vamale as well (see the examples added to Hollyman's list), differences emerge in the subclasses. For example, \textcite{hollyman_etudes_1999} does not mention vowel lengthening, though this is a phenomenon well described for Nêlemwa \parencite[29--33]{bril_nelemwa_2002}. Another possessive noun class not mentioned by Hollyman is that of length shift: \textit{iila}, \textit{il-oong} \textit{ilaa-m} \qu{cauldron, my, your cauldron}. This pattern is described for Bwatoo \parencite[37]{rivierre_bwatoo_2006}, though it seems in every case to be restricted to small groups of nouns. 
\end{sloppypar}

\begin{enumerate}[label=\Alph*.]
	\item inalienable (see set I in \Cref{tab:PossSuffix})
	\item alienable, vowel-final
	\begin{enumerate}[label=B\arabic*.]
		\item -V + suffix: \textit{wata} \qu{digging stick}, \textit{wata-m} \qu{your digging stick}
		\item change of -V + suffix: \textit{da}, \qu{spear} \textit{de-ong} \qu{spear-1\gl{sg}.\gl{poss}} 
		\item[B2a.] Lengthenings are found as well: \textit{hanu}, \qu{picture} \textit{hanuu-ng} \qu{picture-1\gl{sg}.\gl{poss}}
		\item -V + other V + suffix: Not seen in Vamale.
	\end{enumerate}
	\item alienable, consonant-final
	\begin{enumerate}[label=C\arabic*.]
		\item -C is dropped, possessive suffix is added to the rest of the stem. \textit{Xeet} \qu{basket}, \textit{xee-ng} \qu{basket-1\gl{sg}.\gl{poss}}
		\item -C is dropped, last vowel of the stem changes, possessive suffix added to it. Not seen in Vamale, though nouns that used to have a final consonant may have dropped it since.
		\item -C is replaced by an irregular sequence of another consonant and a vowel: \textit{jiket} \qu{arrow} \textit{jike-l-ong/-an} \qu{arrow-1\gl{sg}.\gl{poss}/3\gl{sg}.\gl{poss}}. 
		\item -C is replaced by an irregular vowel. Not seen in Vamale. However, Hollyman's Jawe example \textit{jic}, \textit{jie-n} \qu{belly} is still Vamale \textit{jia-n} \parencite[62]{hollyman_etudes_1999}.
		\item a vowel is introduced between the stem-final consonant and the possessive suffix: \textit{fwaadan} \qu{road}, \textit{fwaadan-i-le} \qu{road-3\gl{pl}}, but also all forms in set Ib.
	\end{enumerate}
\end{enumerate}


\begin{sidewaystable}
	\small
	%\caption{Possessive suffixes and examples}
	\caption{Possessive classes}
% 	\tabcolsep=0.11cm
	%\begin{tabular}{p{1.5cm}p{2cm}p{2.5cm}|p{1.5cm}p{2cm}p{2cm}|p{2cm}|p{2cm}}
		\begin{tabularx}{\linewidth}{QQQ QQQQQ}
		\lsptoprule
		\multicolumn{3}{c}{{Inalienable}}&\multicolumn{5}{c}{{Alienable}}\\\cmidrule(lr){1-3}\cmidrule(lr){4-8}
		\multicolumn{3}{c}{-n}&\multicolumn{3}{c}{C-final stem}& \multicolumn{2}{c}{V-final}\\\cmidrule(lr){1-3}\cmidrule(lr){4-6}\cmidrule(lr){7-8}
		&&&&&&& irregular\\\cmidrule(lr){8-8}
		-V1n/-V2ng& -V:n/-V:ng& ka-n/k-ong& -t/-l-&
		%-C/-ung/ -iile&
		%-p/-v-an&
		-C/-C-ong/-C-a-n&
		-C/-C-eong&
		-V-n-eong\slash\mbox{-go}\slash\mbox{-ea}&-V/-V:-ng\\\midrule
		\textit{si-n, s-ung} `hand'&
		\textit{hnyanaa-n, hnyanaa-ng} `breath'&
		\textit{vwaseeka-n, vwaseek-ong} `sadness'&
		\textit{fedat, fedal-ong} `blood'&
		%mulip, mulivan 'life'&
		\textit{wang, wang-ong} `boat'&
		\textit{hneeng, hneeng-eong} `law'&
		\textit{jo, jo-n-eong} `chicken'&
		\textit{udo, udo-ong} `drink'\\\addlinespace
		\textit{xha-n, xh-ong} `leg'&
		\textit{wîî-n, wîî-ng} `strength'&
		\textit{saleka-n, salek-ong} \qu{possession}& \textit{wadat, wadal-ong} `gun'& \textit{xhetham}, \textit{xhetham-ong} `plate'& \textit{vap, vap-eong} `hunt'& \textit{vuki, vuki-n-eong} `reason, fault'& \textit{hanu, hanu-ung, hanuu} `picture'\\\addlinespace
		\textit{xhapun-an}, \textit{xhapun-ale} `colour' &
		\textit{waa-n, waa-ng} `root'&&
		\textit{vaset, vasel-ale} `swamp clam'&
		\textit{vadang, vadang-ong} `cabin, shelter'&
		\textit{xhanyip, xhanyip-eong} `dream' &
		\textit{mwa, mwa-n-eong} `house'&
		\textit{xa, xaa-ng} ‘tuber cutting for replanting'\\
		\lspbottomrule
	\end{tabularx}
\label{tab:poss}
\end{sidewaystable}

\Cref{tab:poss} shows several things, most of which only apply to paradigm I. Stems ending in /i/ and /u/ cause a progressive assimilation to /u/ in the first person singular (\textit{si-}, \textit{su-ng} \qu{hand, my hand}, \textit{hanu-ung} \qu{my picture}), as described in \sectref{ssec:fronting_u}. Long vowels in the stems of monosyllabic, paradigm I items assimilate the vowel {[ɔ]} of \textit{-ong} 1\gl{sg} (\textit{hnyanaa-ng} \qu{my breath}). Long vowels in the stems of alienable, polysyllabic items lose their length in the possessed form, and a vowel of the possessive morpheme is lengthened (\textit{iila, il-oong} \qu{pot, my pot}, \textit{fwaadan, fwadanuung} \qu{path, my path}). There are alienable items, again with paradigm I forms, where the stem-final /t/ changes to /l/ in possessive contexts. This is due to a Proto-Oceanic liquid that is preserved intervocalically as /l/ in Vamale, but merged with /t/ in coda positions (see \sectref{sec:consphonemes} for more details on finals). The pair \textit{mulip, muliv-an} \qu{life, s/he is alive} is not included in this table because it represents a very small class (the only other confirmed case is \textit{vap\slash vavi} \qu{go on a hunt\slash hunt something}); they probably have a similar background. Alienable forms ending in other consonants add a probably epenthetic \textit{-a-}: \textit{thin} \qu{closing}, \textit{thin-an} \qu{lid}. Those nouns also use direct forms of the Ib set.
Inalienable nouns belong to the following classes: \textit{ka-n}, -V-\textit{ng}, vowel change. 

%However, the alternative possessive form \textit{-gaa} &\qu{1\gl{pl}.\gl{incl}.\gl{poss}} is also attested. \textit{cama bwa me li=mamangaa voilà} (vamale-170908-basket, 00:08:49-00:08:52 ,T=Marie,B=529567,E=532238) 

%\todo{where to put? anyway \textit{gaa} is the free pronoun, \textit{gase} is the bound one}


\begin{table}
	\caption{Possessive suffixes, \textsc{obj} and -\textsc{s\textsubscript{p}}}
	\begin{tabular}{ll ccccc}
		%	&&&\multirow{3}{}{Possession}&&&\\
		\lsptoprule
		&& I& Ib& II & -\textsc{s\textsubscript{p}} &\gl{obj}\\\midrule
\gl{sg} &	1&	\textit{-ng}&	\textit{-ong}&	\textit{-eong} &\textit{ -ong} & \textit{ -eo}\\
		&	2&	\textit{-m}&	\textit{-am}&	\textit{-go}&\textit{-go} & \textit{-ko}\\
		&	3&	\textit{-n}&	\textit{-an}&	\textit{-ea} & \textit{-(e)a} & \textit{-}a \\
		\midrule
\gl{du} &	1\gl{incl}&\textit{	-ju}&	\textit{-aju}&\textit{-ju} &\textit{-gaeu/-gasu} &\textit{-kaeu}\\
		&	1\gl{excl}&	\textit{-bu}&	\textit{-abu}&	\textit{-bu}& \textit{-gabu}&\textit{-kabu}\\
		&	2&	\textit{-u}&	\textit{-au}&	\textit{-gau} &\textit{-gau} & \textit{-kau}\\
		&	3&	\textit{-lu}&	\textit{-alu}&	\textit{-lu }&\textit{-lu} & \textit{-lu} \\
		\midrule
\gl{pl} &	1\gl{incl}&	\textit{-j}e&\textit{-aje}&	\textit{-je} &\textit{-gaa} & \textit{-kaa}\\
		&	1\gl{excl}&	\textit{-be}&	\textit{-abe}&	\textit{-be} & \textit{-abe}&\textit{-kabe} \\
		&	2&	\textit{-vwe}&	\textit{-avwe}&	\textit{-vwe} &  \textit{-gavwe}& \textit{-kavwe}\\
		&	3&\textit{	-le}&\textit{	-ale}&	\textit{-le} & \textit{-le }&\textit{ -le}\\
		\lspbottomrule
	\end{tabular}
\label{tab:CompSuffix}
\end{table}

Anything that is not usually possessed (\textit{vap} \qu{hunt}) or is a loanword (\textit{teeriko}, \textit{teerikoneong} \qu{(my) shirt}), is possessed with set II suffixes. An epenthetic \textit{n} appears if following a morpheme-final vowel. Diachronically, it seems likely that this \textit{-n} was a linker morpheme descended from POc *na \parencite[234]{lynch_historical_2000}, probably an independent word (i.e. not a clitic), followed by the possessor pronoun or noun. It would have become a construct suffix over time. The pronouns were incorporated into the possessum later on. This would also explain the forms \textit{-eo(ng)} \qu{1\gl{sg}.\gl{poss}} and \textit{-ea} \qu{3\gl{sg}.\gl{poss}}: the free pronouns are /jo/ and /ja/ to this day, and forms like /ɣaju/ \qu{male} can be pronounced /ɣaeu/, which suggests that glides can be realized as more open vowels in some contexts.
This means that the epenthetic \textit{-e-} found in IIb, \gl{obj}, and -\gl{S\textsubscript{P}} suffixes does not seem to be phonologically conditioned like in Caac: `\textit{e} \qu{\gl{ind}} is used when the lexeme it follows ends with a consonant (18, 19) while \textit{le} \qu{\gl{ind}} is utilized when the lexeme it follows ends with a vowel (16, 17)' \parencite[32]{cauchard_study_2014}.

Some words have two possessive paradigms, one with set I suffixes, like \textit{i mulip} \qu{the life}\slash \textit{mulivong} \qu{I am alive}, and another with set II forms, i.e. -\textit{eong}, \textit{mulip-eong} \qu{my life}. Speakers disagree on whether the latter form is more emphatic and marked, i.e. \qu{my life} vs \qu{this life of mine}, or whether there is a meaning difference. This same discussion arises with other nouns as well, e.g. \textit{wat-ong} or \textit{wata-n-eong} \qu{the digging stick which is mine (and nobody else's)}.

%, \textit{pelap} (&\qu{type of mat}, is only used to describe a mat, if ever the mat is possessed, it is \textit{xam} &\qu{mat}). \textit{Waan} &\qu{root} does not follow this, but could be used metaphorically for humans often enough to warrant its special status?
%-\textit{gaa}, the inclusive 1PL possessive form, is refused for many words to the profit of the exclusive form -\textit{je}. \textit{In maan} &\qu{skin (of human, possibly from \textit{atemaan}, &\qu{face})}: in maaje, pala "home", \textit{faati} &\qu{language}, \textit{gana} &\qu{colour like X}, \textit{xhetham} &\qu{plate}. The inclusive/exclusive distinction is recognised, but ignored.



%-t/-lan might be to differentiate -t final morphemes from others in possession, but I have not found a minimal pair so far.

\subsection{Alienable}
\label{ssec:AlPoss}
\is{Possession!Alienable}

Alienable nouns form the bulk of Vamale nouns. An open class which seems to be slowly gaining members from the inalienable class, its possessive suffixes are mostly from Set Ib or II. Those nouns inflected with Set Ib forms, usually associated with inalienable nouns, are often semantically close to inalienable nouns, such as certain kinship terms, or things belonging to bodies (spirit, breath, tail). One major difference from inalienable forms is the fact that they never drop their final consonant. 

%\ea
%
%\langinfo{}{}{} KG:491
%\gll a cana ka th=e bwa vee mwa vwaseekan mwa ko \textbf{vukin-eong} mwa
%
% \gl{expl} vagina \gl{cnj} \gl{ass}=1\gl{sg} \gl{ipfv} fuck \gl{rep} sad \gl{rep} because cause-1\gl{sg}.\gl{poss} \gl{rep}
%
%\glt \qu{Ah shit, I had just fucked up then, I was sorry [for them] because it was my fault}
%
%
%\z

\subsection{Inalienable}
\label{ssec:InalPoss}
\is{Possession!Inalienable}

Vamale has a considerable number of nouns which must be possessed. If the possessor is unknown, inalienable nouns take a generic \textit{-n} \qu{\gl{nspec}}. \is{Specificity! Generic \textit{-n}} 
Inalienably possessed nouns form a closed class, bearing paradigm I suffixes in \Cref{tab:PossSuffix}. The only seeming exception is the nominalizations bearing \textit{=ka-n}, but note that \textit{=ka-} is a grammatical word which can be omitted from the nominalizing constructions (see \sectref{kan}), and takes paradigm II suffixes. \textit{Ka-} is inalienable in the sense that the construction it precedes must be possessed. The locative nouns (``prepositions") mentioned in \sectref{ssec:WCPrepoNouns} are members of this closed class.
Inalienable nouns use both \textit{-n} \qu{3\gl{sg}.\gl{poss}} and \textit{-m} \qu{2\gl{sg}.\gl{poss}} for quotation forms.
Some inalienable nouns, in compounds where they are not the head, lose their possessive morphology when they do not have a specific referent, e.g. \textit{mwa-n nyama} \qu{glasses (in general, nobody's glasses)}. If possessed, however, it is the second part of the compound that is possessed, i.e. \textit{mwa-n nyamaa-ng} \qu{my glasses}. See also \textit{vwa suki(-n)} \qu{pay (for something)}, which, nominalized, becomes \textit{xavwasuki} \qu{money-spender}, glossed in (\ref{ex:xavwasuki}). This is not attested for \textit{e-vwadi ya-n} \qu{thumb (lit. \gl{nmlz}.\gl{ins}-peel.with.fingers starchy.food-\gl{poss})}, possibly because a thumb is itself an inalienable concept, whereas glasses are alienable.
	
	\ea \label{ex:xavwasuki}
	\gll xa=vwa-suki\\
	 \gl{nmlz}.\gl{agt}=do-price\\
	\glt \qu{a money-spender}
	\z 

Some nouns are inalienable, but cannot be possessed by humans. They thus do not take any personal possessive suffixes, although they otherwise follow classical inalienable morphology, i.e. an alternation between generic \textit{-n} \qu{\gl{nspec}}, specific \textit{-n} \qu{3\gl{sg}.\gl{poss}}, and a postponed possessor noun phrase. Examples include \textit{maa-n} \qu{point, visible side}, \textit{thin-an} \qu{lid} (derived from \textit{thin}, \qu{close}), \textit{xhii-n} \qu{fin},\footnote{Animal anatomy terms have probably lost some ground since the culture mostly abandoned sustenance fishing, but even then there are remarkably few animal-specific body terms. \textit{Uba-n} \qu{fish scale}, \textit{thaang-an} \qu{tentacle} and \textit{jahlo} \qu{rooster's crest} are the only other terms recorded in the lexicon. Animal anatomy, like plant anatomy, is described in the same terms as their human equivalent.} and \textit{vaa-n} \qu{undergarment, base}, which must be followed by what garment covers it, shoes or pants or a dress. Consider \Cref{tab:bit}. The nouns listed in the table need a (specified or implicit) bigger context, which is usually postponed as a modifier. They are part of a part-whole relationship that ties specific part-of-a-whole words to their possessor entity, e.g. \textit{bati} \qu{long piece of wood that is detached from the tree}, \textit{xada-n} \qu{(jagged) detached part of something hard}, while others are more generic, such as \textit{xhula-n} \qu{extremity, consequence}. 

\begin{table}
	\caption{Parts of things}
	\begin{tabular}{ll}
	\lsptoprule
		\textit{thin-an}& \qu{lid}\\
		\textit{xhii-n}& \qu{fin}\\
		\textit{vaa-n}& \qu{undergarment, base}\\
		\textit{bala-n} & piece of something long (rope, stick)\\
		\textit{hmanya-n} & crumbs of wood or stone\\
		\textit{xada-n} & shard, sharp-edged bit\\
		\textit{bati} & detached bit of wood\\\addlinespace
		\textit{bate} &  extremity, beginning/end of an entity\\
		\textit{xhula-n} & consequence, extremity of event \\
		\textit{maa-n}& \qu{point, visible side}\\
	\lspbottomrule
	\end{tabular}
	\label{tab:bit}
\end{table}
%\todo{Do you consider these as classifiers ? do they just not express part-whole relationship ? }
 
\section{Classifiers}
 \label{sec:CL}
 \is{Classifiers}
Classifiers are a well-established and rich class in both Nêlêmwa \parencite{bril_nelemwa_2002} and Iaai, but not thought to be widespread in Mainland New Caledonian. In Vamale, there is a semantically defined group of nouns that easily and often forms quasi-possessive phrases with other nouns. Most of these nouns are inalienably possessed. They form the head of their phrase; the other noun cannot bear an article (\ref{ex:foodCL2}), and, in the cases discussed here, cannot occur without the head (in the semantic contexts which warrant these constructions). In any case, the nouns discussed here can occur without the modifying noun. Following \textcite{aikhenvald_classifiers_2000}, this study will call these nouns classifiers. Words like \textit{saleka-n} \qu{possession}, \textit{coola-n} \qu{task, part of collective work}, \textit{sana-} \qu{content}, \textit{san-fe} (content-take) \qu{hunting bounty}, \textit{mwa-n} \qu{container}, as well as the items in \Cref{tab:bit}, work the same way, with the exception that they can be omitted. The latter group thus seems to be frequent compound heads, described as generic-specific constructions \parencite[86]{aikhenvald_classifiers_2000}, rather than classifiers. They are discussed in detail in \sectref{sec:CompN}.

%Another frequent way of describing complex concepts is with possessive noun phrases. The syntactic head will correspond to a modified noun, i.e. the possessum. There is considerable overlap with compounds, with the difference that both parts of the possessive phrase can be specific.
 
 \subsection{Relational classifiers: Food}
 \is{Classifiers!Relational classifiers}
 \label{ssec:food_CL}
 The members of this subgroup are all linked to special verbs (see \Cref{tab:cl_verb}) and cannot be omitted in favor of the modifying noun (i.e. the substance consumed). They are all inalienably possessed, and the substance they classify is invariably alienably possessed. 
 

 \ea\label{ex:foodCL1} 
 \gll Na li=vataan \textit{xhua-m} (juu-mani)\\
  \gl{dem} \gl{def}.\gl{pl}=various proteiny.food-2\gl{sg}.\gl{poss} sacred-bird\\
 \glt \qu{These are your various dishes (of wood pigeons).}
 \z 
 
 \ea\label{ex:foodCL2}
 \gll na li=vataan (*i) juu-mani\\
  \gl{dem} \gl{def}.\gl{pl}=various \gl{def}.\gl{sg} sacred-bird\\
 \glt \qu{These are the various (live, or inedible) wood pigeons.} (\emph{not}: These are your pigeons to eat)
  \z
 
 \begin{table}
 	\centering
 	\caption{Classifiers, corresponding verbs, and corresponding food item}
 	\begin{tabular}{lll}
	\lsptoprule
 		Classifier & Verb & Food item\\\midrule
 		\textit{xhua-} & \textit{xhwi} & \qu{(proteiny) food}\\
 		\textit{fwaa-} & \textit{fwai} & \qu{chewy food} (e.g. magnagna root)\\
 		\textit{xhuta-} & \textit{xhuti} & \qu{scrunchy food} (e.g. sugar cane)\\
 		\textit{u-} & \textit{xaje} & \qu{juicy food} (fruit, vegetables)\\
 		\textit{ya-}& \textit{xhajake}& \qu{starchy food} (tubers, rice, bread)\\
 		\midrule
 		\textit{fatoo-}&\textit{fato} &\qu{hot drink}\\
 		\textit{udoo-} & \textit{udu} &\qu{cold drink}\\
	\lspbottomrule
 	\end{tabular}
 \label{tab:cl_verb}
 \end{table}

%The kinship term is almost always used, expressing the parental link of the person spoken to and oneself, or, if absent, the person spoken about and oneself.&\qu{grandfather X, how are you today?}. This explains why people do not use kinship terms when speaking about people to me. Children however will be taught their relations by people saying ``your uncle X came by yesterday".

  
 \subsection{Relational classifier \textit{ka}}
 \label{ssec:ka_CL}
 \is{ka!Lexically assigned linker \textit{ka}@\textit{ka}}

 Vamale has a morpheme \textit{ka-} that takes inalienable possessive morphology, is used to mark usually unpossessed nouns as possessed (\ref{ex:kan2}), and contains semantic information about the relationship between possessor and possessum. The morpheme is obligatory in certain scenarios but optional in others. I call \textit{ka-} a relational classifier:\footnote{Following \textcite[136]{aikhenvald_classifiers_2000} and especially \textcite[399]{lichtenberk_oceanic_2009}.} \textit{ka} does not specify the nature of either noun phrase in the possessive phrase.  However, the semantics of the classifier is somewhat vague and could be described as ``relating to the possessor", a term borrowed from the gloss for the relational classifier \textit{'e} in Boumaa Fijian \parencite[135]{dixon_grammar_1988}. While \textcite{lichtenberk_oceanic_2009} accepts such aberrant behavior for a classifier on the grounds that languages have unique categories, he mostly uses the term ``possessive marker". 
 Another, perhaps simpler analysis would see \textit{ka-} as a linker, following \textcite{bril_ownership_2012} among others: linking the head noun to its modifier, \textit{ka} is semantically vague and closer to the head than to the dependent.
 \textit{ka-} is obligatory for \textit{daahma} \qu{chief} (\ref{ex:kaposs1}), \textit{phwêêdi} \qu{youngest child/sibling}, \textit{bifidu} \qu{twin} and a few other nouns possessed through interpersonal relationships. The linker is in some cases part of a lexicalized possessive noun phrase (\ref{ex:kaposs2}, \ref{ex:kan1}). It is also found on: 
 
 \begin{itemize}
 \sloppy
 \item \textit{udee} \qu{medication}, to introduce the ailment to be cured, e.g.\textit{udee ka-n nyaabu} \qu{medicine against mosquitoes}
 \item \textit{juuju ka-m} \qu{your truth, you're right}
 \item \textit{xhwata} \qu{baldness, bald head} e.g. \textit{xhwata ka-m} \qu{your bald head, you are bald}
 \end{itemize}
 
 Contrary to the relational classifiers in \sectref{ssec:food_CL}, \textit{ka-} cannot be used anaphorically. %It seems likely that \textit{ka} is a reflex of POc \textit{*ña} \parencite[136]{dixon_grammar_1988}. 
 Note that this lexically assigned, obligatory linker is distinct from the optional \textit{ka-} that can be added to any nominalization, which was described in \sectref{kan}. Apart from the form, the two morphemes share the alternation of the initial \textit{k-} with nasals and non-velar plosives. If the stem ends in these consonants, /k/ is dropped: /kan/ $\rightarrow$  /an/ / N,p,t,l,c\_\_. Given the similarity in shape and function, I suggest that the two are related.%\todo{Can you develop this ?}
  

 \ea\label{ex:kan2}
 \gll difaadi ka-n\\
  echo \gl{rel}.\gl{clf}-\gl{nspec}\\ 
 \glt \qu{its echo} {[vamale-180727-elicitation-ganadd-1]}
 \z
 
 \ea\label{ex:kaposs1}
 \gll  daahma k-ong\\ 
  chief \gl{rel}.\gl{clf}-1\gl{sg}.\gl{poss}\\ 
 \glt \qu{my chief}
 \z
 
 
 
 \ea\label{ex:kaposs2}
 \gll daahma ka-n mani\\ 
  chief \gl{rel}.\gl{clf}-\gl{nspec} bird\\ 
 \glt \qu{chief of birds [\textit{erythrura psittacea}]}
 \z
 


\ea\label{ex:kan1}
\gll i ka-n xavwaxhan\\
 louse \gl{rel}.\gl{clf}-\gl{nspec} dog\\
\glt \qu{flea}
 \z


 There are a number of irregular forms which show etymological final consonants that have since merged to the plosives \textit{-t, -p, -k, -c} (see \sectref{sec:consphonemes}). For example, \textit{fedat} \qu{blood}, POc *\textit{daaR}, retains the historical liquid in the possessed form \textit{feda-l-am} \qu{your blood}. The possessive morphology of these irregular forms follows the same laws (compare (\ref{ex:kaposs2}) to (\ref{ex:fedat})) and is a predictable allomorph of \textit{ka-}. In fact \textit{ka-} only follows vowel-final possessums, whereas consonant-final words take \textit{-an}, e.g. \textit{japit}/\textit{japit-an} \qu{travel provisions; salary}. Prosodically, constructions with \textit{ka-} have at least two p-words: the possessed NP, and \textit{ka}, which can hence be analyzed as an anticlitic: not an own grammatical word (g-word), but an own phonological word (p-word) \parencite{zuniga_anti_2014}. However, the consonant-final possessed NPs only have one main stress, and thus count as a single p-word. \textit{Fedalan}, for example, is split [ˈfɛⁿ.da.lan], with the stress on the first syllable (see vamale-181020-01-batis-bonjour-tontons-1, 01:27). We thus have a situation where the same morpheme has a different phonological status depending on its host's final form.  It seems likely that the indirect possessive constructions with the linker \textit{ka-} being an own g-word was the original situation, and that the linker was phonologically incorporated into the host for most contexts: \textit{juujuu ka-m} (truth \gl{poss}-2\gl{sg}.\gl{poss}) \qu{you're right (not \qu{your truth})}, but \textit{muliv-ong} (life 1\gl{sg}.\gl{poss}) \qu{I am alive}. Related to this, at least diachronically, are the possessive morphemes discussed in \sectref{kan}, though the latter are optional. % (vs \textit{mulip eong} {my life})
 
 \ea \label{ex:fedat}
 % \langinfo{}{}{} 
 \gll i=a e-fii-kaa i=in-maa-n apuli ka i=fedala-n apuli \\ % ka ni xhwat si-be nyako i bosu kanabwe
  \gl{def}.\gl{sg}=\gl{rel} \gl{recp}-sew-1\gl{pl}.\gl{incl} \gl{def}.\gl{sg}=skin-face-\gl{nspec} person \gl{cnj} \gl{def}.\gl{sg}=blood-\gl{nspec} person \\ % \gl{conj} \gl{def}.\gl{pl}=little 
 \glt \qu{What ties us together is the human skin and the human blood.} {[}2018 enterrement coutume présentation 1:28]
 \z
 
 \begin{sloppypar}
 An ambiguous case is that of \textit{mae} \qu{fire, light}. Normally non-possessed, two possessive constructions are employed to talk about \textit{mae} \qu{lighter}, probably calqued from a local French term \textit{feu} \qu{fire; lighter} (\ref{ex:maekong}). One is that of alienable nouns (e.g. \textit{-n-eong}), and the other uses \textit{ka}. Since there is a choice in the morphology to be used, \textit{mae} is reminiscent of another relational morpheme: the optional, focussed possession marker \textit{ka}, discussed in the following section.
 \end{sloppypar}


	\ea \label{ex:maekong}
	\gll mae k-ong\\	
	 fire \gl{clf}.\gl{poss}-1\gl{sg}\\	
	\glt \qu{my lighter (optional: \emph{my} lighter)}
	\z
	
	\ea
	\gll mae-n-eong\\	
	 fire-\gl{poss}-1\gl{sg}\\
	\glt \qu{my lighter, my fire, my light}
	\z 


\subsection{Focussed Possession marker \textit{ka}}
\label{ssec:foc_poss_ka}
\is{ka!Focussed Possession marker \textit{ka}@\textit{ka}}
\textit{ka} is an optional preposition marking a possessor as focussed, and/or the possessum as especially important.\footnote{Something similar is described as ``close possession" in Hebrew \parencite{berman_modern_1978}.} Because it does not say anything about the nature of the possessor itself, I do not call it a possessor classifier \parencite[125]{aikhenvald_classifiers_2000}. Instead, it seems more reasonable to call it a linker like the other \textit{ka} forms, as \textit{ka-} applies to alienable nouns and can be replaced by more conventional possessor marking \parencite[136]{aikhenvald_classifiers_2000}. Contrary to the lexically assigned obligatory linker discussed in \sectref{ssec:ka_CL}, this \textit{ka} is optional, i.e. the noun can be marked as possessed by \textit{-n} \qu{\gl{poss}} instead (\ref{ex:thala_xhaohmu}). Furthermore, while the obligatory classifier uses inalienable possessive suffixes, e.g. \textit{daahma ka-n/k-ong} \qu{chief \gl{poss}-3\gl{sg}.\gl{poss}/\gl{poss}-1\gl{sg}.\gl{poss}}, \textit{phwêêdi k-an/k-ong} \qu{youngest child}, \textit{ka} \qu{\gl{foc}.\gl{poss}} takes nominal and pronominal possessors. Since the two morphemes are probably related and are identical in form and position, there is inter-speaker variation in their distribution: Jacob Oué maintains \textit{thala k-ong} \qu{my knife} where Jean-Philippe Oué, about 20 years younger, uses \textit{thala ka yo} (2019-08-05 JP ka:33). However, Philippe Gohoupe, born in the 1940s, uses \textit{ka} with the pronoun \textit{gavwe} instead of the suffix \textit{-vwe} in (\ref{ex:ka gavwe}), i.e. he uses \textit{ka} \qu{\gl{foc}.\gl{poss}} like Jean-Philippe Oué.  %and the animate possessor of a tool or a dynamic relationship in established ones.

\ea
\label{ex:thala_xhaohmu}
\gll 	thala-n i=xhaohmu	\\
	knife-\gl{poss} \gl{art}.\gl{sg}=elder	\\
\glt  \qu{the elder's knife}		
\z

\ea
\gll 	thala ka i=xhaohmu	\\
	knife \gl{foc}.\gl{poss} \gl{art}.\gl{sg}=elder	\\
\glt  \qu{the \textit{elder's} knife} \\ \qu{the knife belonging to the elder through his use of it}
\z

Note the ambiguity between the 3rd person possessive \textit{-n} shown in \textit{daahma ka-n} \qu{chief \gl{clf}.\gl{poss}-3\gl{sg}.\gl{poss}} and the anaphoric \textit{-n} in (\ref{ex:ka_nspec}), which I take as grounds to differentiate the two \textit{ka}. \textit{Udee k-ong} \qu{medicine \gl{poss}-1\gl{sg}.\gl{poss}} is not attested.

\ea \label{ex:ka_nspec}
% \langinfo{}{}{} 
\gll cip=e=caihna-n hapi udee ka-n i=da hê \\
 \gl{neg}=1\gl{sg}=know-\gl{nspec} \gl{comp} medicine \gl{foc}.\gl{poss}-\gl{ana} \gl{def}.\gl{sg}=what yes \\
\glt \qu{I don't know what the medicine for it [mosquitoes] is, yeah.} {[AG1:212]}
\z

Beneficiaries, but not direct objects, can be focussed on with \textit{ka} \qu{\gl{foc}.\gl{poss}}, because the beneficiary constructions (\sectref{sec:oblique}) derive from \textit{si-} \qu{hand} and \textit{ko-} \qu{on}, both taking possessors. 

\ea \label{ex:ka gavwe}
\gll e=hole-ke nyasi-vwe ka=gavwe\\
 1\gl{sg}=thank-\gl{tr} for-2\gl{pl} \gl{foc}.\gl{poss}=2\gl{pl}\\
\glt \qu{I thank you (in particular) [for what you did?].}
\z

 %Interestingly, these forms, although alienable in the sense that their quotation form does not bear possessive morphology, take paradigm I suffixes, i.e. what is associated with inalienable forms. The same is true for the forms possessed with the relational classifier \textit{ka-} (see \Cref{ssec:ka_CL}). Since the latter nouns' roots all end on vowels, and the former all on consonants, it is likely that this was historically the same morpheme, though the assimilated forms such as \textit{fedal-am} &\qu{your blood} or \textit{muliv-am} &\qu{your life} are now integrated phonologically into the noun, and are suffixes.
 
 \subsection{Noun classifiers}
 \label{ssec:NounCL}
 \is{Classifiers!Noun classifiers}
 
 Noun classifiers, using Aikhenvald's term and definition \parencite[81]{aikhenvald_classifiers_2000}, are assigned based on semantics. Not every noun in Vamale takes a noun classifier (in fact, only plant species do). A plant species can take different noun classifiers, depending on the meaning intended. Similarly to relational classifiers, they can be used anaphorically, and indeed usually are \parencite[87]{aikhenvald_classifiers_2000} for a discussion of how typical this is). These noun classifiers are alienably possessed, but rarely occur without their possessive suffix \textit{-n} for reasons tied to their semantic nature: the possessor tends to be generic. \textit{Yee} \qu{tree}, and possibly \textit{doo-n} \qu{leaf} are exceptions to this tendency. \textit{Xhaapwe} \qu{fruit} occurs in the same environments as the noun classifiers shown in \Cref{tab:Noun_Class}, but is not possessed with \textit{-n}.
 
 \begin{table}
 	\centering
 	\caption{Noun classifiers in Vamale}
 	
 	\begin{tabular}{ll}
 		\lsptoprule
 		Form & Gloss\\\midrule
 		\textit{doo-n }& \qu{leaf} \\
 		\textit{i-n} & \qu{bark}\\
 		\textit{vuki-n}&\qu{stem} \\
 		\textit{ye(e)-n}&\qu{tree} \\
 		\textit{muu-n} & \qu{blossom} \\
 		\textit{si-n} & \qu{living branch}\\
 		\lspbottomrule
 	\end{tabular}
 \label{tab:Noun_Class}
 \end{table}

Words for trees are always formed in the same way: \textit{yee} \qu{wood, tree}, followed by the species of the plant, e.g. \textit{yee-n sep} \qu{tree-\gl{poss} coco}. The same goes for fruit (\textit{xhaapwe sep}), leaves (\textit{doo-n sep}), and bark (\textit{i-n sep}). The word for the plant species alone denotes an abstract referent. %Note that \textit{xhaapwe} &\qu{fruit}, while the head of the compound, is not a possessum.  

\section{Compound nouns}
\label{sec:CompN}
\is{Nouns!Compound Nouns}

Compound nouns are nouns with a nominal head and modifier (of verbal or nominal nature). Both noun-on-noun and verb-on-noun compounds may be exocentric, i.e. describing a referent not mentioned in the compound (e.g. \textit{vaci nyu} \qu{kernel/nucleus fish} \qu{anchor}), or endocentric, where a clue to the referent is present (e.g. \textit{we jati} \qu{water salt/sea} \qu{seawater}). Similar constructions with verbal heads are discussed in \sectref{ssec:CompV}. 

Other ways of modifying a noun are relative clauses, and possessive constructions. While these also result in a noun phrase which acts as a single constituent (\ref{ex:cloud}), compound nouns are single words: they do not tolerate lexical insertions and often have idiosyncratic meanings (e.g. \textit{hmape-thoatit} and \textit{yeen-bwan} in ex. \ref{ex:cloud}). Compound nouns may be exocentric, use metaphors to describe their referent, or harbor other semantic relations not found in noun phrases with a relative clause, e.g. part-of-whole ones. However, in many cases there are no phonological, morphological, or syntactic ways to clearly differentiate compound nouns from unmarked noun - relative clause constructions. 

Indeed, stress is no distinguishing indicator, as both the elements of complex nouns as well as those of noun phrases are stressed like single words (e.g. \textit{apuli Teganpaik} [ˈapuˌli ˌtʰegãnˈpaːik] \qu{Teganpaik resident} and \textit{hmape-thoatit} [ˈm̥ãpe ˈtʰɔ.a.tit] \qu{cloud}), a phenomenon also attested in Nêlêmwa \parencite[204]{bril_noms_2004}. The prosodic domain above the word-level seems to stress the modifier over the modified, again regardless of the syntactic nature of the construction.

\ea \label{ex:cloud}
% \langinfo{}{}{} 
\gll a=xaleke {\ob\ob\ob}hmape-thoatit{\cb} a fiing{\cb} a kon nyasipoke{\cb} nya-pwa-n yee-n-bwan\\
 3\gl{sg}=see flesh-sky \gl{rel} dark \gl{rel} \gl{prog} gather \gl{loc}-on-\gl{nspec} stick-\gl{poss}-mountain\\
\glt \qu{She saw dark clouds gathering on the mountain tops.} {[GC:13]}
\z

\begin{sloppypar}
Many former possessive noun phrases have become lexicalized into compounds and denote a single referent, e.g. \textit{mwa-n nyama} \qu{glasses (lit. container-\gl{poss} eye)}.\footnote{Interestingly, \textit{nyamaa-} is inalienable and would usually carry a possessive suffix. It is also shorter in the compound than in its possessed form.} Similarly, some noun phrases have been lexicalized that are formed with \textit{ko-n} \qu{on-\gl{nspec}}, discussed in \sectref{ssec:ko_N}. \textit{Ko} is otherwise productive to express part-of-whole relationships: compare established, opaque \textit{bucit kon xhan} \qu{joint? on leg}\qu{ankle} to transparent \textit{fubuun ko-n uvu} \qu{heap on-\gl{nspec} yam} \qu{heap of yam}, and newly coined \textit{chambre-à-air ko-n velo} \qu{air chamber on-\gl{nspec} bicycle} \qu{bicycle tyre}. Both the constructions with possessive \textit{-n} and those featuring the preposition represent a middle ground between noun phrases and compound nouns, as they are not single words phonologically, and are transparently derived. This grammar arbitrarily calls the complex nouns which still bear possessive morphology ``compound nouns" proper, as they constitute the majority of forms, and the ones without such traces, e.g. the ones in \Cref{tab:vaci}, ``bare compound nouns". There does not seem to be a semantic logic behind which words form bare compounds and which bear possessive morphology. While the distribution hinges mostly on the modified noun, there are nouns showing both patterns (e.g. \textit{fwa-n bua-n} \qu{hole-\gl{poss} ?-3\gl{sg}.\gl{poss}} \qu{navel} but \textit{fwa thâ-n} \qu{hole excrement-3\gl{sg}.\gl{poss}} \qu{anus}). Nor is alienability of the head noun a criterion, compare inalienable \textit{nyivwa-n goakan} \qu{mouth-\gl{poss} middle} \qu{window} to alienable \textit{vaaya-n goakan} \qu{movement-\gl{poss} middle} \qu{see-saw movement, rocking movement}. A tentative explanation would combine:
\end{sloppypar}

\begin{enumerate}
	\sloppy
	\item lexically defined distribution (i.e. it depends on the word, e.g. \textit{vaci} \qu{hard bit, nucleus})
	\item avoiding ambiguity with similar-looking words (\textit{we} \qu{water} never takes \textit{-n} in compounds because \textit{wee-n} \qu{sap, fluid} already exists)
	\item lexicalization leads in some cases to the loss of \textit{-n} (perhaps \textit{fwa-thâ-n} \qu{anus})
\end{enumerate}

Compound nouns can be classified via several dimensions: semantic properties (endocentric vs exocentric, and the semantic relationship between the components), morphological properties (i.e. whether possessive morphology is present within the compound, as discussed above), possessive strategies (i.e. how a compound is possessed by an outside participant), and word-classes represented. 

Possessive morphology docks onto the right border of the compound, except if the first noun is inalienable, and the second part not a noun, e.g. \textit{vabu-ng thamo} \qu{grandchild-1\gl{sg}.\gl{poss} woman} \qu{my granddaughter}. Regarding word-classes, the main types are N+N, N+V, and V+N. In some cases, compounds integrate yet another compound, which yields more complex forms, e.g. \textit{{\ob}mwa{\cb} {\ob}cabi {\ob}vai-vun{\cb\cb}} \qu{house smash stone-blue/green} \qu{prison} (colonial prisons employed forced labor). Nominal compounds containing adverbs are present in related languages \parencite[200]{bril_noms_2004} and have semantic equivalents in Vamale, but are not necessarily compounds, syntactically speaking (see \sectref{sec:n+adv}). The intensifier \textit{juu} \qu{very, real, sacred} is a very common part of compounds as well, e.g. \textit{juu mwa} \qu{traditional house}, \textit{juu apuli} \qu{Kanak}, \textit{juu toot} \qu{thatching straw} etc.

\subsection{N + N compounds}
\is{Nouns!Compound Nouns!Noun on noun}
Noun-on-noun compounds include many examples of a semantically vague head which is followed by a modifying noun with a more precise or specific meaning. Contrary to a possessive construction, the resulting compound cannot dispense with the modifier, and would lose its meaning entirely if the head stood alone. For example, when talking about sewing (\textit{sili}), one could not use \textit{vaci} \qu{nucleus, most important part} and expect people to immediately grasp that one is talking about the thread (\textit{vaci sili}). See Tables~\ref{tab:vaci} and~\ref{tab:maan} for lists of compounds affected by this. 

\begin{table}
	\caption{Compounds with \textit{vaci} \qu{nucleus, most important part}}
	\begin{tabular}{lll}
		\lsptoprule
		Form & Meaning second morpheme & Meaning of compound \\\midrule
		\textit{vaci nyu}& \qu{fish}& \qu{anchor} \\
		\textit{vaci nyima-n} & \qu{heart-3\gl{sg}.\gl{poss}} & \qu{darling} \\
		\textit{vaci xayu} & \qu{male}& \qu{little boy} \\
		\textit{vaci uvu}& \qu{yam}& \qu{yam tuber} \\
		\textit{vaci nyivwa-n} & \qu{mouth-3\gl{sg}.\gl{poss}}& \qu{tooth}\\
		\textit{vaci bwa-n}& \qu{head-3\gl{sg}.\gl{poss}} &\qu{his cranial box (round part)}  \\
		\textit{vaci sili}&  \qu{sew}& \qu{sewing thread} \\
		\textit{vaci mata}& \qu{sing} &\qu{musical theme} \\
		\textit{vaci vua}&  \qu{net}& \qu{net sinker}\\
		\lspbottomrule
	\end{tabular}
	\label{tab:vaci}
\end{table}

\begin{table}
	\caption{Compounds with \textit{maan} \qu{face, tip}}
	\begin{tabular}{lll}
		\lsptoprule
		Form & \multicolumn{2}{c}{Meaning of} \\\cmidrule(lr){2-3}
		     & other morpheme & compound \\
		\midrule
		\textit{maan hmeewan} & \qu{sand} & \qu{tip of a sandbank }\\
		\textit{maan op} & \qu{(high) tide} & `waves touching the shore, \\
						 &					& \quad tip of the tide' \\
		\textit{maan da} & \qu{spear} & \qu{spear tip} \\
		%	\textit{buudi li=maan} & shedding \gl{def}.\gl{pl}=& \item
		\textit{cu-pwan maan}  &\qu{standing-on} & \qu{stand in front of something} \\
		\textit{fwa-n maan vua}  &\qu{hole X net} & \qu{net mesh} \\
		\textit{nyau maan} &\qu{bad} & \qu{ugly}\\
		\textit{se maan}  &\qu{one} & \qu{same, to repeat}\\
		\textit{in maan}  &\qu{leather, bark} & \qu{human live skin}\\
		\lspbottomrule
	\end{tabular}
	\label{tab:maan}
\end{table}

%\begin{table}
%	\begin{tabular}{lll}
%		Form & Glosse zweites Morphem & Bedeutung ganze Form\item
%		\lsptoprule
%		hmuun uta & Regen & Nieselregen\item
%		hmuun mae & Feuer & Rauch\item
%		hmuun sikaa & Zigarette & Zigarettenrauch\item
%		hmuun doop & Erde & Staub\item
%		hmuun vai & Stein & Steinstaub\item
%		hmuun we & Wasser & Wasserdampf\item
%		hmuun jati & Meer & Seenebel\item
%		hmuun bwanpu & Land & diesige Luft\item
%	\end{tabular}
%\end{table}

Many complex or abstract concepts are described via compounds, and use metaphors for a part of it: \textit{duu-n we} (bone-\gl{poss} water) \qu{water current}.

\subsubsection{Endocentric N+N compounds}

A special group of endocentric noun-on-noun compounds use modifying nouns like \textit{thamo} \qu{woman, female}, \textit{xayu} \qu{boy, male}, \textit{xhaohmu} \qu{elder, be old} and \textit{xawe} \qu{youth, be young}. These are often predicates, and (at least the latter two) are also attested as stative verbs, e.g. \textit{mani-thamo} \qu{female bird}, or \textit{i thamo-xhaohmu} \qu{the old woman}. The same meaning is achieved with a relative clause, e.g. \textit{i thamo a xhaohmu}. A relative clause consisting of a nominal predicate, a stative or intransitive verb, or with an inanimate or generic subject (meaning the relativizer \textit{a} and \textit{a} \qu{3\gl{sg}} are juxtaposed), may omit the relativizer, especially in fast speech. 
Some compounds contain elements that are found nowhere else, e.g. \textit{thivaan sin} \qu{smallest finger} (\textit{thivaan} is opaque), \textit{bu-cit ko-n xhan} (? on-\gl{nspec} leg) \qu{ankle}, and \textit{bu-vaci xhan} (?-nucleus leg) \qu{ankle (bone?)}. Some metaphorical terms for body pars are listed in \Cref{tab:body_comp}.


\begin{table}
	\caption{Body parts described metaphorically by (the head of a) compound}
	\begin{tabularx}{\textwidth}{llQ}
		\lsptoprule
		Form & \multicolumn{2}{c}{Meaning of} \\\cmidrule(lr){2-3}
			 & other morpheme & of compound \\\midrule
		\textit{futho kon xha-n} &plantain on leg-3\gl{sg}.\gl{poss} &\qu{calf}\\
		 \textit{we-n ma iila} &water-\gl{poss} \gl{com} pot &\qu{part of the sole that does not leave a footprint}\\
		 \textit{vi-n sep} &shell-\gl{poss} coconut &\qu{kneecap}\\
		 \textit{bet ca-n duu-n} &worm in-\gl{nspec} bone-3\gl{sg}.\gl{poss} & \qu{bone marrow}\\
		 \lspbottomrule
	\end{tabularx}
\label{tab:body_comp}
\end{table}



%However, the former construction can be modified with a relative clause (e.g. \textit{i thamo-xhaohmu a xahnang} \qu{the woman-old who is good}), whereas consecutive relative clauses modifying the same noun are not attested, as in (\ref{ex:noRELREL}).

%\ex \label{ex:noRELREL}
%
%\gll paa sinu i xavwakhân a xhaohmu a welo
%
% \gl{pa} ill \gl{def}.\gl{sg}=dog \gl{rel} old \gl{cnj}/\gl{rel} crazy 
%
%\glt \qu{The old and crazy dog has died}
%
%
%\z

\subsubsection{Exocentric N + N compounds}
Exocentric N+N compounds, the smaller of the two noun-on-noun groups, have more or less opaque meanings. They may describe the referent's appearance:

\begin{itemize}
	\item	\textit{bwa-n ibwen} \qu{head-\gl{poss} squid} \qu{a species of deadwood mushroom}
	\item	\textit{ot-an-bwa-n thupila} \qu{belt-\gl{poss}-band-\gl{poss} devil} \qu{`devil's headband', an orange \textit{nyaouli} savannah vine}
	\item \textit{tha-n mutô} \qu{excrement-\gl{poss} sheep} \qu{a species of grass}
\end{itemize}

The compounds may also describe a purpose of the referent:

\begin{itemize}
	\item \textit{thili thâ} \qu{wipe excrement} \qu{a species of grass}
	\item \textit{dipi maphwên} \qu{wrap leftovers} \qu{a species of tree}
	\item \textit{fa-mulip} \qu{\gl{caus}-life} \qu{\textit{plectranthus parviflorus}, a medical plant}
\end{itemize}

\Cref{tab:compounds} lists other terms describing the function, origin, or other association. Not all of them are compounds.

\begin{table}
	\caption{Concepts described by their function, origin, or other associations (e.g. toxicity).}
	\begin{tabularx}{\textwidth}{llQ}
		\lsptoprule
		Form & \multicolumn{2}{c}{Meaning of} \\\cmidrule(lr){2-3}
		 & other morpheme & of compound \\\midrule
		\textit{ye iila} &tree pot &\qu{tree (whose fruit were used as a container)}\\
		 \textit{xhwaeo pupwaale} &taro European &\qu{dry taro (imported by Europeans)}\\
		 \textit{dongan thupila} &orange corpse&\qu{\textit{citrus macroptera} (toxic when raw)}\\
		 \textit{mwa-n suhmee} &container-\gl{poss} spit &\qu{lung}\\
		 \textit{mwa-n gila} &container-\gl{poss} bitter &  \qu{gallbladder}\\
		 \textit{mwa-n nyai-n} &container-\gl{poss} child-3\gl{sg}.\gl{poss} &\qu{uterus}\\
		 \textit{fwa-thâ-n} &hole-excrement-3\gl{sg}.\gl{poss} &\qu{anus}\\
		 \textit{xa-funa} &\gl{agt}.\gl{nmlz}-preach &\qu{middle finger} \\ 
		 \textit{ape-tha-xhuuni} &\gl{nmlz}-throw-spear sling &\qu{index finger}\\
		 \lspbottomrule
	\end{tabularx}
 \label{tab:compounds}
\end{table}

\subsubsection{Compounds with two heads}
\is{Nouns!Compound Nouns!Compounds with two heads}
Additive noun-on-noun compounds, where the sense depends on both, equal elements, are rare, but exist, e.g. \textit{bween phwê} \qu{night month} \qu{date (specific day decided upon)}.\footnote{\textit{bwen} \qu{night} is lengthened, a hint at its possessum origin, see \textit{iila}, \textit{iloo-ng} \qu{cauldron, my cauldron} in \sectref{sec:Poss}.} %\parencite[203]{bril_noms_2004} 

\subsection{The question of Noun + Adverb compounds}
\label{sec:n+adv}
While noun-and-adverb compounds were described for other languages of the area \parencite[192, 200]{bril_noms_2004}, this work could not find any which were distinguishable from noun phrases that are modified with an adverb, as the latter's position is identical in both cases, and prosody is the same in compounds as it is in complex noun phrases. One distinguishing feature of other noun compounds includes the use of words that have otherwise fallen out of use, a lack of possessive morphology, or an unusual word order. None of this was found with adjectives modifying nouns. Furthermore, noun phrases containing an adverb can be modified by a relative clause (\ref{ex:n+adv}). However, since relative clauses modify single-word nouns as well as noun phrases (to which an adverb can belong), no convincing syntactic arguments seem to posit the existence of said compounds. 

\ea \label{ex:n+adv}
\gll li=xhaohmu habu\textsuperscript{adv} a vwa wada-le\\
 \gl{def}.\gl{pl}=elder long.ago \gl{rel} \gl{exist} gun-3\gl{pl}.\gl{poss}\\
\glt \qu{the elders of yore who had guns}
\z



\subsection{N + V compounds}
\is{Nouns!Compound Nouns!Noun on verb}
A major group of compound nouns featuring verbal elements put the noun first. The noun is then described by the verb, which denotes a property, state, or function of the noun. 

\begin{itemize}
	\item \textit{mwa-n vwa-ila} \qu{house-\gl{poss} do-pot} \qu{cooking house, i.e. kitchen}
	\item \textit{mwa-n sohmun} \qu{house-\gl{poss} study} \qu{school}
	\item \textit{tii siteke} \qu{notch, writing sacred} \qu{the Bible}
\end{itemize}

Stative verbs in general tend to signify properties or states, and the nominal part of the compound usually refers to the bearer of these properties: \textit{we nyam} \qu{water sweet} \qu{sweetwater}.
While almost all compounds contain intransitive verbs, I found one exception (\ref{ex:fwantiti}). The verb here has a similar function to the intransitive verbs described above, i.e. it assigns a property to the noun.

\ea \label{ex:fwantiti}
\gll fwa-n titii-ke\\
 hole-\gl{poss} be.wet-\gl{tr}\\
\glt \qu{moist spot, buried spring}
\z

\subsection{V + N compounds}
\is{Nouns!Compound Nouns!Verb on noun}
\begin{sloppypar}
While the majority of nominal compounds featuring verbs put the nominal head first, another group put the noun second. Many of these are derived verb phrases, like the endocentric metonymic compound \textit{vun muun} \qu{blue/green flower} (a species name for a blue flower), the exocentric word for humpback, \textit{xhwe duun} \qu{twisted back}, or \textit{fun aman} \qu{wilt something} \qu{dry season}.
\end{sloppypar}

%xx \textit{fe nyamaa-n} \qu{take eye-3\gl{sg}.\gl{poss}} \qu{catch the attention of}, where the possessor of inalienable \textit{nyamaan} \qu{eye} is the undergoer and Experiencer, and the subject is the Stimulus.
%\ex
%
%\langinfo{}{}{} GP:60
%\gll tha fe nyamaa-ng i puudo
%
% \gl{ass} take eye-1\gl{sg}
%
%\glt \qu{I noticed the whale}
%
%
%\z

Others are more opaque, e.g. the metaphor \textit{vun bwan-toot} \qu{blue grasstips} \qu{blue hour, briefly before nightfall}. Consider the exocentric compounds naming the days of the week:\footnote{The week is called \textit{da(wee)n vwa siteke} \qu{between prayers}, itself an exocentric compound derived from a prepositional phrase.} Monday to Thursday count the days passed since Sunday (see \Cref{tab:week}). The word for Friday, \textit{fa-siit}, is likely derived from the Christian taboo of eating meat on that day: the causative prefix \textit{fa-} docks into \textit{siit}, likely related to \textit{sitoon} \qu{taboo} and \textit{siteke} \qu{sacred, forbidden}.

\begin{table}
	\caption{Days of the week}
	\begin{tabular}{lll}
	\lsptoprule
		Sunday & \textit{vwa siteke} & \qu{do sacred, pray}\\
		Monday& \textit{se vwa-siteke}& \qu{one [day after] Sunday}\\
		Tuesday&\textit{thaloo vwa-siteke} &\qu{two Sunday}\\
		Wednesday& \textit{thiien vwa-siteke}& \qu{three Sunday}\\
		Thursday&\textit{fava vwa-siteke}& \qu{four Sunday}\\
		Friday&\textit{fa-siit}& \qu{\gl{caus}-?}\\
		Saturday&\textit{savato}& (from \textit{sabbat})\\
	\lspbottomrule
	\end{tabular}
\label{tab:week}
\end{table}

Other exocentric nominal compounds also include \textit{vwa} \qu{do; \gl{exist}}:
\begin{itemize}
\item \textit{vwa det} \qu{make rustling sound} \qu{dead coral bits on a beach, or as a floor covering}
\item \textit{vwa jinun} \qu{\gl{exist} magical power} \qu{sorcerer, magician}
\item \textit{vwa wii-an} \qu{\gl{exist} field-3\gl{sg}.\gl{poss}} \qu{shaved head}
\end{itemize}

This chapter has covered simple nouns, their syntactically relevant semantic features and how possession works. After discussing complex nouns, many of which stem from noun phrases, the account shall now move on to noun phrases proper.

%A group of compounds is ambiguous with respect to their wordclass. \textit{xahnang nyima-n} \qu{good heart-3\gl{sg}.\gl{poss}} \qu{s/he is happy} is a verb phrase with nominal morphology and an idiosyncratic meaning. It is used as a predicate and never as a 
